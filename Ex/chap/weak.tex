\chapter{Phänomenologie der schwachen Wechselwirkung}
Die schwache Wechselwirkung ist für den Zerfall von Quarks und Leptonen verantwortlich.
Es sind keine gebundenen Zustände bekannt, die sich aufgrund der schwachen Wechselwirkung bilden.
In Streuexperimenten ist die schwache Wechselwirkung nur schwer beobachtbar, da alle Teilchenreaktionen, die der schwachen Wechselwirkung unterliegen, sehr geringe Wirkungsquerschnitte haben.

\section{Leptonen}
Wie bereits in den vorigen Kapiteln gesehen, gibt es \textbf{3 Leptonenfamilien}:
\begin{equation*}
	\twovec{\nu_e}{\el}\quad\twovec{\nu_\mu}{\mu^-}\quad\twovec{\nu_\tau}{\tau^-}.
\end{equation*}
Wie die Quarks unterscheiden sich auch die geladenen Leptonen deutlich in ihrer Masse.

\section{Typen der schwachen Wechselwirkung}
Es gibt zwei Typen der schwachen Wechselwirkung, nämlich die Wechselwirkung mit einem
\begin{itemize}
	\item \textbf{neutralen Strom} unter Austausch eines $Z^0$-Bosons, der die Flavors der Quarks und die Leptonen unverändert lässt und Wechselwirkungen mit einem
	\item \textbf{geladenen Strom} unter Austausch eines $W^+$- oder $W^-$-Bosons, der flavorändernd wirken kann.
\end{itemize}

Die erwähnten Austauschbosonen sind Vektorbosonen (Spin-1-Teilchen).

\section{Geladene Ströme}
Man kann geladene Ströme in 3 Klassen unterteilen: in \textit{leptonische}, \textit{semileptonische} und \textit{nichtleptonische Prozesse}.

\subsection{Leptonische Prozesse}
Wenn das W-Boson nur an Leptonen koppelt, spricht man von \textit{leptonischen Prozessen}:
\begin{equation*}
	l + \bar{\nu}_l \longleftrightarrow l' + \bar{\nu}_{l'}.
\end{equation*}
Beispiele:
\begin{align*}
	\tau^- &\rightarrow \mu^- + \bar{\nu}_\mu + \nu_\tau \\
	\tau^- &\rightarrow \el + \bar{\nu}_e + \nu_\tau.
\end{align*}

\subsection{Semileptonische Prozesse}
Bei \textit{semileptonischen Prozessen} koppelt das W-Boson sowohl an Quarks, als auch Leptonen:
\begin{equation*}
	q_1 + \bar{q}_2 \longleftrightarrow l + \bar{nu}_l.
\end{equation*}
Das wohl bekannteste Beispiel hierfür ist der $\upbeta$-Zerfall des Neutrons.

\subsection{Nichtleptonische Prozesse}
Diese Prozesse finden ohne Beteiligung von Leptonen statt:
\begin{equation*}
	q_1 + \bar{q}_2 \longleftrightarrow q_3 + \bar{q}_4.
\end{equation*}
Ein Beispiel für einen solchen Prozess ist der $K^+$-Zerfall in zwei Pionen:
\begin{equation*}
	K^+ \rightarrow \pi^0 + \pi^+.
\end{equation*}

\section{Universalität der schwachen Wechselwirkung}
Anhand der Zerfallsbreiten des $\tau^-$-Zerfalls lässt sich zeigen, dass die schwache Wechselwirkung an Quarks und Leptonen mit der gleichen Kopplungskonstanten $g$ koppelt.
Die Hauptzerfallskanäle sind
\begin{align*}
	\tau^- &\rightarrow \mu^- + \bar{\nu}_\mu + \nu_\tau \\
	\tau^- &\rightarrow \el + \bar{\nu}_e + \nu_\tau \\
	\tau^- &\rightarrow \bar{u} + d + \nu_\tau
\end{align*}
mit den angenommenen Zerfallsbreiten $\Gamma_{\tau e} \approx\Gamma_{\tau \mu}$ und $\Gamma_{\tau d\bar{u}}\approx 3\cdot\Gamma_{\tau e}$ (Faktor 3 kommt aus der Farbentartung).
Man erwartet also für die Lebensdauer
\begin{equation*}
	\tau_\tau = \frac{\hbar}{\Gamma_{\tau e} + \Gamma_{\tau \mu} + \Gamma_{\tau d\bar{u}}} \approx \SI{3.1e-13}{\second},
\end{equation*}
was in guter Übereinstimmung mit dem experimentellen Wert von
\begin{equation*}
	\tau_\tau^\text{exp} = \SI{2.906(10)e-13}{\second}
\end{equation*}
ist.

Dies ist kein besonders stichhaltiger Beweis für die Universalität der schwachen Wechselwirkung, da die Lebensdauer sehr stark von der Masse des $\tau$ abhängig ist,
allerdings wird die Annahme aus semileptonischen Zerfällen der Hadronen weiter gestützt.

\section{Der Cabbibo-Winkel}
\begin{figure}
	%\includegraphics{/path/to/figure} TODO
	\caption{Feynmangraphen für die Bestimmung des Cabbibo-Winkels}
	\label{fig:cabbibo}
\end{figure}

Wie bereits bekannt werden die Quarks in Familien nach Masse und Ladung unterteilt:
\begin{equation*}
	\twovec{u}{d}\quad\twovec{c}{s}\quad\twovec{t}{b}.
\end{equation*}
Man beobachtet nun aber neben Übergängen innerhalb einer Familie auch Übergänge zwischen den Familien in einem geringen Maße.
Bezüglich der geladenen Ströme der schwachen Wechselwirkung ist daher der Partner des Flavoreigenzustandes $\ket{u}$ nicht $\ket{d}$, sondern eine Linearkombination von $\ket{d}$ und $\ket{s}$, welche $\ket{d'}$ genannt werden soll.
Entsprechende Überlegungen lassen sich für die übrigen Flavoreigenzustände machen.
Man kann diese Linearkombinationen nun durch den \textbf{Cabbibo-Winkel} wie folgt ausdrücken:
\begin{equation*}
	\ket{d'} = \cos\theta_\text{C}\ket{d} + \sin\theta_\text{C}\ket{s}.
\end{equation*}
Experimentell wird dieser Cabbibo-Winkel durch die hadronischen Zerfälle und deren Verzweigungsverhältnisse bestimmt (vgl. \autoref{fig:cabbibo}).
