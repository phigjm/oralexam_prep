\chapter{Phänomenologie der schwachen Wechselwirkung}
Die schwache Wechselwirkung ist für den Zerfall von Quarks und Leptonen verantwortlich.
Es sind keine gebundenen Zustände bekannt, die sich aufgrund der schwachen Wechselwirkung bilden.
In Streuexperimenten ist die schwache Wechselwirkung nur schwer beobachtbar, da alle Teilchenreaktionen, die der schwachen Wechselwirkung unterliegen, sehr geringe Wirkungsquerschnitte haben.

\section{Leptonen}
Wie bereits in den vorigen Kapiteln gesehen, gibt es \textbf{3 Leptonenfamilien}:
\begin{equation*}
	\twovec{\nu_e}{\el}\quad\twovec{\nu_\mu}{\mu^-}\quad\twovec{\nu_\tau}{\tau^-}.
\end{equation*}
Wie die Quarks unterscheiden sich auch die geladenen Leptonen deutlich in ihrer Masse.

\section{Typen der schwachen Wechselwirkung}
Es gibt zwei Typen der schwachen Wechselwirkung, nämlich die Wechselwirkung mit einem
\begin{itemize}
	\item \textbf{neutralen Strom} unter Austausch eines $Z^0$-Bosons, der die Flavors der Quarks und die Leptonen unverändert lässt und Wechselwirkungen mit einem
	\item \textbf{geladenen Strom} unter Austausch eines $W^+$- oder $W^-$-Bosons, der flavorändernd wirken kann.
\end{itemize}

Die erwähnten Austauschbosonen sind Vektorbosonen (Spin-1-Teilchen).

\section{Geladene Ströme}
Man kann geladene Ströme in 3 Klassen unterteilen: in \textit{leptonische}, \textit{semileptonische} und \textit{nichtleptonische Prozesse}.
