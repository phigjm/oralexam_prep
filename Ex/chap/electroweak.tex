\chapter{Elektroschwache Theorie}

\section{Reele W- und Z-Bosonen}
In $\el\pos$-Collidern ist zur Erzeugung von $Z^0$-Bosonen gemäß
\begin{equation*}
	\el + \pos \rightarrow Z^0
\end{equation*}
eine Schwerpunktsenergie von $\sqrt{s}=M_\text{Z}c^2$ erforderlich.
W-Bosonen können aufgrund der Ladungserhaltung nur in Paaren erzeugt werden
\begin{equation*}
	\el + \pos \rightarrow W^+ + W^-,
\end{equation*}
weshalb auch höhere Schwerpunktsenergien nötig sind.

Für viele Jahre bestand die einzige Möglichkeit für die Erzeugung der Bosonen darin,
die Quarks im Proton auszunutzen, gemäß
\begin{alignat*}{2}
	u + \bar{u} &\rightarrow Z^0 \quad d + \bar{u} &\rightarrow W^- \\
	d + \bar{d} &\rightarrow Z^0 \quad u + \bar{d} &\rightarrow W^+.
\end{alignat*}
Anstatt aber bloß zwei Protonen frontal kollidieren zu lassen und auf die Anti-Seequarks zu hoffen (relativer Anteil im Proton im Mittel $\approx 0.04$),
lässt man lieber Protonen und Antiprotonen kollidieren, wodurch man deutlich weniger Schwerpunktsenergie im Mittel benötigt.

Die Bosonen können dann durch die Zerfälle
\begin{alignat*}{2}
	Z^0 &\rightarrow \el\pos \quad W^+ &\rightarrow \pos + \nu_\text{e} \\
	Z^0 &\rightarrow \mu^- \mu^+ \quad W^+ &\rightarrow \mu^+ + \nu_\mu .
\end{alignat*}
Experimentell beobachtet man also ein hochenergetisches $\el\pos$- oder $\mu^-\mu^+$-Paar, wobei Lepton und Antilepton in entgegengesetzten Richtungen wegfliegen müssen.
Die Neutrinos können hierbei allerdings nicht direkt nachgewiesen werden, allerdings kann man aus der Summe der Transversalimpulse den fehlenden Impuls dem Neutrino zuschreiben.

\section{Zerfälle des W-Bosons}
Wir haben gesehen, dass W-Bosonen nur an linkshändige Fermionen koppelt und zwar an alle mit gleicher Stärke.
Dabei führt die Cabbibo-Rotation lediglich zu einer kleinen Korrektur bezüglich der Kopplung an die Quarks.
Wenn jedoch wirklich eine Universalität in der schwachen Wechselwirkung existiert,
dann müssten im Zerfall des $W^+$ die Paare $\pos\nu_e, \mu^+\nu_\mu, \tau^+\nu_\tau, u\bar{d'}$ und $c\bar{s'}$ im Verhältnis $1:1:1:3:3$ erzeugt werden.
Eine Messung bestätigt diese Ergebnisse.

\section{Zerfälle des Z-Bosons}
Vermittelt das Z-Boson nun die schwache Wechselwirkung in gleicher Weise wie das W-Boson,
dann sollten obige Überlegungen auch für die Zerfälle des Z-Bosons gelten.
Messungen ergeben allerdings, dass die Wahrscheinlichkeit für den Zerfall in geladene Leptonen und Neutrinos deutlich unterschiedlich sind.
Die Kopplung des Z-Bosons ist somit offensichtlicherweise auch von der elektrischen Ladung abhängig.
Demnach ist das Z-Boson nicht einfach nur ein ungeladenes W-Boson, sondern vermittelt eine kompliziertere Wechselwirkung.

\section{Die elektroschwache Vereinheitlichung}
Die Eigenschaften des Z-Bosons können mit einer Theorie ästhetisch beschrieben werden,
die die elektromagnetische und schwache Wechselwirkung als zwei Aspekte einer gemeinsamen Wechselwirkung auffasst, der \textit{Theorie der elektroschwachen Wechselwirkung}.

\subsection{Schwacher Isospin}
\begin{figure}
	\centering
	\includegraphics[width=.7\textwidth]{./img/weakisospintable.pdf}
	\caption{Multipletts der elektroschwachen Wechselwirkung}
	\label{fig:weakisospin}
\end{figure}
Die elektroschwache WW lässt sich elegant formulieren, indem man eine neue Quantenzahl, den \textit{schwachen Isospin} $T$ einführt.
Linkshändige Fermionen werden dabei als Dupletts angeordnet, die sich durch Emission/Absorption von W-Bosonen ineinander umwandeln können.
Rechtshändige Fermionen werden in Singuletts angeordnet, da die W-Bosonen an sie nicht koppeln.
Tabelle \ref{fig:weakisospin} fasst diesen Zusammenhang zusammen.
Um den Formalismus konsequent weiterzuführen, muss das $W^-$-Boson die Quantenzahl $T_3=-1$ und das $W^+$-Boson die Quantenzahl $T_3=+1$ haben, damit der schwache Isospin eine Erhaltungsgröße ist.
Um das Triplett aufzufüllen, muss dann noch ein Zustand $T=1, T_3=0$ existieren, welchen wir $W^0$ nennen.
Dieser Zustand koppelt mit der gleichen Stärke an die linkshändigen Fermiondupletts wie die $W^\pm$-Bosonen.
$W^0$ kann nicht mit $Z^0$ identisch sein, da die Kopplung des $Z^0$ noch von der Ladung abhängig ist.
Dieses Problem wird dadurch gelöst, dass man einen weiteren Zustand $B^0$ postuliert, der das Singulett des schwachen Isospins sein soll ($T=0, T_3=0$).
Die Kopplungsstärke dieses Zustandes muss nicht mit der des Tripletts übereinstimmen, man nenne die Stärke $g'$.

\subsection{Der Weinbergwinkel}
\begin{figure}
	\centering
	\includegraphics[width=.7\textwidth]{./img/diffsigmaelectroweak.pdf}
	\caption{Differenzieller Wirkungsquerschnitt $\diff[\sigma]{Q^2}$ für Reaktionen mit neutralen und geladenen Strömen in der tiefinelastischen $e^\pm p$-Streuung}
	\label{fig:diffsigmaelectroweak}
\end{figure}
Experimentell kennen wir in der Tat zwei neutrale Vektorbosonen: das Photon und das $Z^0$.
Die Idee der elektroschwachen Theorie: wir drücken das Photon und $Z^0$ als zueinander orthogonale Linearkombinationen von $B^0$ und $W^0$ aus,
ganz analog zur Quarkmischung mit dem Cabbibo-Winkel gilt dann
\begin{alignat*}{3}
	\ket{\gamma} &= \cos\theta_\text{W}\ket{B^0} &&+ \sin\theta_\text{W}\ket{W^0} \\
	\ket{Z^0} &= -\sin\theta_\text{W}\ket{B^0} &&+ \cos\theta_\text{W}\ket{W^0}.
\end{alignat*}
Aus der Forderung, dass das Photon an die Ladung der links- und rechtshändigen Fermionen koppelt, nicht aber in die Neutrinos erhält man den Zusammenhang zwischen den schwachen Ladungen $g$ und $g'$
\begin{gather*}
	\tan\theta_\text{W} = \frac{g'}{g}\quad\sin\theta_\text{W}=\frac{g'}{\sqrt{g'^2 + g^2}}\quad\cos\theta_\text{W}=\frac{g}{\sqrt{g'^2 + g^2}}.
\end{gather*}
Für die elektrische Ladung dann analog
\begin{equation*}
	q_e = g\cdot\sin\theta_\text{W}.
\end{equation*}

Durch Messungen des Weinbergwinkels erhält man, dass die schwache Kopplungskonstante ($\alpha_\text{s}\propto g\cdot g$) demnach etwa viermal stärker als die elektromagnetische ($\alpha\propto e\cdot e$) ist.
Der Grund für die geringe effektive Stärke der schwachen Wechselwirkung (bei kleinen Energien) liegt im Propagatorterm $p$ in den Feynmanamplituden der Reaktionen.
Bei diesem geht die Masse der Austauschteilchen im Nenner ein, weswegen für ein masseloses Photon dieser Beitrag im Nenner nicht vorhanden ist:
\begin{equation*}
	\mathcal{M}\propto p\propto \frac{1}{Q^2 + m^2_\text{boson}} =
	\begin{cases}
		\frac{1}{Q^2},\ \text{EM} \\
		\frac{1}{Q^2 + m^2}<\frac{1}{q^2},\ \text{weak}
	\end{cases}
\end{equation*}

Im Limes hoher Energien $Q^2\rightarrow\infty$ wird der Beitrag der Bosonenmassen im Propagator aber vernachlässigbar klein.
Deswegen sind die Unterschiede für Wirkungsquerschnitte bei geladenen und neutralen Strömen bei großen Energien verschwindend gering, wie Messungen in \autoref{fig:diffsigmaelectroweak} bestätigen.
Die schwache und elektromagnetische Wechselwirkung vereinen sich.

\subsection{Zahl der Neutrinosorten}
Das detaillierte Studium der Erzeugung von $Z^0$-Bosonen durch $\el\pos$-Paarvernichtung ermöglicht eine sehr genaue Überprüfung der Vorhersagen der elektroschwachen Vereinheitlichung.
Es müssen für die totale Zerfallsbreite bei diesem Prozess alle partialen Breiten summiert werden.
Die Partialbreite für $\nu\bar{\nu}$-Paare beträgt unter Beachtung der verschiedenen Chiralitätszustände (rechtshändige Neutrinos treten in der Natur nicht auf $\rightarrow \hat{g}_\text{R} = 0 = T_3$)
\begin{equation*}
	\Gamma_\nu \approx \SI{165.8}{\MeV}.
\end{equation*}
Die Quarkbeiträge ergeben unter Beachtung der Farbentartung
\begin{equation*}
	\Gamma_\text{d} = \Gamma_\text{s} = \Gamma_\text{b} = 3\cdot\SI{122.4}{\MeV}.
\end{equation*}
Entsprechend für die u- und c-Quarks
\begin{equation*}
	\Gamma_\text{u} = \Gamma_\text{c} = 3\cdot\SI{94.9}{\MeV}.
\end{equation*}
Dabei wird zwischen up-artigen und down-artigen Quarks unterschieden, da diese verschiedene schwache Isospins aufweisen und damit verschieden an das $Z^0$ koppeln.
Der Beitrag der geladenen Leptonen ergibt
\begin{equation*}
	\Gamma_\text{e} = \Gamma_\mu = \Gamma_\tau = \SI{83.3}{\MeV}.
\end{equation*}

Somit beträgt die gesamte Zerfallsbreite (nach Strahlungskorrekturen)
\begin{equation*}
	\Gamma_\text{tot} = \SI{2497(6)}{\MeV},
\end{equation*}
was sich ausgezeichnet mit dem experimentellen Wert
\begin{equation*}
	\Gamma_\text{tot,ex} = \SI{2495(2)}{\MeV}
\end{equation*}
deckt.

Gäbe es noch eine weitere Neutrinosorte, so wäre die totale Zerfallsbreite um \SI{166}{\MeV} größer.
Man kann also aus diesem Resultat schließen, dass es genau 3 Sorten leichter Neutrinos gibt.
