\chapter{Elektroschwache Theorie}

\section{Reele W- und Z-Bosonen}
In $\el\pos$-Collidern ist zur Erzeugung von $Z^0$-Bosonen gemäß
\begin{equation*}
	\el + \pos \rightarrow Z^0
\end{equation*}
eine Schwerpunktsenergie von $\sqrt{s}=M_\text{Z}c^2$ erforderlich.
W-Bosonen können aufgrund der Ladungserhaltung nur in Paaren erzeugt werden
\begin{equation*}
	\el + \pos \rightarrow W^+ + W^-,
\end{equation*}
weshalb auch höhere Schwerpunktsenergien nötig sind.

Für viele Jahre bestand die einzige Möglichkeit für die Erzeugung der Bosonen darin,
die Quarks im Proton auszunutzen, gemäß
\begin{alignat*}{2}
	u + \bar{u} &\rightarrow Z^0 \quad d + \bar{u} &\rightarrow W^- \\
	d + \bar{d} &\rightarrow Z^0 \quad u + \bar{d} &\rightarrow W^+.
\end{alignat*}
Anstatt aber bloß zwei Protonen frontal kollidieren zu lassen und auf die Anti-Seequarks zu hoffen (relativer Anteil im Proton im Mittel $\approx 0.04$),
lässt man lieber Protonen und Antiprotonen kollidieren, wodurch man deutlich weniger Schwerpunktsenergie im Mittel benötigt.

Die Bosonen können dann durch die Zerfälle
\begin{alignat*}{2}
	Z^0 &\rightarrow \el\pos \quad W^+ &\rightarrow \pos + \nu_\text{e} \\
	Z^0 &\rightarrow \mu^- \mu^+ \quad W^+ &\rightarrow \mu^+ + \nu_\mu .
\end{alignat*}
Experimentell beobachtet man also ein hochenergetisches $\el\pos$- oder $\mu^-\mu^+$-Paar, wobei Lepton und Antilepton in entgegengesetzten Richtungen wegfliegen müssen.
Die Neutrinos können hierbei allerdings nicht direkt nachgewiesen werden, allerdings kann man aus der Summe der Transversalimpulse den fehlenden Impuls dem Neutrino zuschreiben.

\section{Zerfälle des W-Bosons}
Wir haben gesehen, dass W-Bosonen nur an linkshändige Fermionen koppelt und zwar an alle mit gleicher Stärke.
Dabei führt die Cabbibo-Rotation lediglich zu einer kleinen Korrektur bezüglich der Kopplung an die Quarks.
Wenn jedoch wirklich eine Universalität in der schwachen Wechselwirkung existiert,
dann müssten im Zerfall des $W^+$ die Paare $\pos\nu_e, \mu^+\nu_\mu, \tau^+\nu_\tau, u\bar{d'}$ und $c\bar{s'}$ im Verhältnis $1:1:1:3:3$ erzeugt werden.
Eine Messung bestätigt diese Ergebnisse.

\section{Zerfälle des Z-Bosons}
Vermittelt das Z-Boson nun die schwache Wechselwirkung in gleicher Weise wie das W-Boson,
dann sollten obige Überlegungen auch für die Zerfälle des Z-Bosons gelten.
Messungen ergeben allerdings, dass die Wahrscheinlichkeit für den Zerfall in geladene Leptonen und Neutrinos deutlich unterschiedlich sind.
Die Kopplung des Z-Bosons ist somit offensichtlicherweise auch von der elektrischen Ladung abhängig.
Demnach ist das Z-Boson nicht einfach nur ein ungeladenes W-Boson, sondern vermittelt eine kompliziertere Wechselwirkung.

\section{Die elektroschwache Vereinheitlichung}
Die Eigenschaften des Z-Bosons können mit einer Theorie ästhetisch beschrieben werden,
die die elektromagnetische und schwache Wechselwirkung als zwei Aspekte einer gemeinsamen Wechselwirkung auffasst, der \textit{Theorie der elektroschwachen Wechselwirkung}.

\subsection{Schwacher Isospin}
\begin{figure}
	\centering
	%\includegraphics[width=.7\textwidth]{./img/weakisospintable.pdf} TODO
	\caption{Multipletts der elektroschwachen Wechselwirkung}
	\label{fig:weakisospin}
\end{figure}
Die elektroschwache WW lässt sich elegant formulieren, indem man eine neue Quatenzahl, den \textit{schwachen Isospin} $T$ einführt.
Linkshändige Fermionen werden dabei als Dupletts angeordnet, die sich durch Emission/Absorption von W-Bosonen ineinander umwandeln können.
Rechtshändige Fermionen werden in Singuletts angeordnet, da die W-Bosonen an sie nicht koppeln.
Tabelle \ref{fig:weakisospin} fasst diesen Zusammenhang zusammen.
Um den Formalismus konsequent weiterzuführen, muss das $W^-$-Boson die Quantenzahl $T_3=-1$ und das $W^+$-Boson die Quantenzahl $T_3=+1$ haben, damit der schwache Isospin eine Erhaltungsgröße ist.
Um das Triplett aufzufüllen, muss dann noch ein Zustand $T=1, T_3=0$ existieren, welchen wir $W^0$ nennen.
Dieser Zustand koppelt mit der gleichen Stärke an die linkshändigen Fermiondupletts wie die $W^\pm$-Bosonen.
$W^0$ kann nicht mit $Z^0$ identisch sein, da die Kopplung des $Z^0$ noch von der Ladung abhängig ist.
Dieses Problem wird dadurch gelöst, dass man einen weiteren Zustand $B^0$ postuliert, der das Singulett des schwachen Isospins sein soll ($T=0, T_3=0$).
Die Kopplungsstärke dieses Zustandes muss nicht mit der des Tripletts übereinstimmen, man nenne die Stärke $g'$.

\subsection{Der Weinbergwinkel}
