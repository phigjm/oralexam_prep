\chcapter{Elektroschwache Theorie}

\section{Reele W- und Z-Bosonen}
In $\el\pos$-Collidern ist zur Erzeugung von $Z^0$-Bosonen gemäß
\begin{equation*}
	\el + \pos \rightarrow Z^0
\end{equation*}
eine Schwerpunktsenergie von $\sqrt{s}=M_\text{Z}c^2$ erforderlich.
W-Bosonen können aufgrund der Ladungserhaltung nur in Paaren erzeugt werden
\begin{equation*}
	\el + \pos \rightarrow W^+ + W^-,
\end{equation*}
weshalb auch höhere Schwerpunktsenergien nötig sind.

Für viele Jahre bestand die einzige Möglichkeit für die Erzeugung der Bosonen darin,
die Quarks im Proton auszunutzen, gemäß
\begin{alignat*}{2}
	u + \bar{u} &\rightarrow Z^0 \quad d + \bar{u} &\rightarrow W^- \\
	d + \bar{d} &\rightarrow Z^0 \quad u + \bar{d} &\rightarrow W^+.
\end{alignat*}
Anstatt aber bloß zwei Protonen frontal kollidieren zu lassen und auf die Anti-Seequarks zu hoffen (relativer Anteil im Proton im Mittel $\approx 0.04$),
lässt man lieber Protonen und Antiprotonen kollidieren, wodurch man deutlich weniger Schwerpunktsenergie im Mittel benötigt.

Die Bosonen können dann durch die Zerfälle
\begin{alignat*}{2}
	Z^0 &\rightarrow \el\pos \quad W^+ \rightarrow \pos + \nu_\text{e} \\
	Z^0 &\rightarrow \mu^-\mu^ \quad W^+ \rightarrow \mu^+ + \nu_\text{\mu}.
\end{alignat*}
Experimentell beobachtet man also ein hochenergetisches $\el\pos$- oder $\mu^-\mu^+$-Paar, wobei Lepton und Antilepton in entgegengesetzten Richtungen wegfliegen müssen.
