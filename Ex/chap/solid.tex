\chapter{Festkörperphysik}

\section{Konstruktion der Wigner-Seitz-Zelle}
Einen Gitterpunkt wählen und Verbindungsstrecken zu sämtlichen anderen Gitterpunkten durch Normalebenen halbieren.
Eingegrenztes Gebiet ist dann Wigner-Seitz-Zelle.

\section{Das reziproke Gitter}
Das zu einem Ortsraumgitter reziproke Gitter ist definiert durch
\begin{equation*}
	\mvec{K} = h\mvec{A_1} + k\mvec{A_2} + l\mvec{A_3}\quad h,k,l\in\mathbb{Z}.
\end{equation*}
Dabei muss die Relation
\begin{equation*}
	\mvec{A_i}\mvec{a_j} = 2\pi\delta_{ij}
\end{equation*}
erfüllt sein.
Man kann die Vektoren $\mvec{A_i}$ durch die Konstruktionsvorschrift
\begin{equation*}
	\mvec{A_i} = 2\pi\varepsilon_{ijk}\frac{\mvec{a_k}\times\mvec{a_k}}{\mvec{a_1}\cdot(\mvec{a_2}\times\mvec{a_3})}.
\end{equation*}

\section{Beugung am Gitter}
Damit eine Beugung auftritt muss gelten
\begin{equation*}
	\Delta\mvec{k} = \mvec{K},
\end{equation*}
mit $\Delta\mvec{k}=\mvec{k}-\mvec{k'}$, dem Differenzwellenvektor der einfallenden Welle und $\mvec{K}$ einem reziproken Gittervektor.

Multiplizieren wir von links mit $\mvec{a_i}$, dann erhalten wir die \textbf{Laue-Bedingung}
\begin{align*}
	\mvec{a_1}\cdot\Delta\mvec{k} &= 2\pi h \\
	\mvec{a_2}\cdot\Delta\mvec{k} &= 2\pi k \\
	\mvec{a_3}\cdot\Delta\mvec{k} &= 2\pi l.
\end{align*}

Andere Form:
\begin{equation*}
	\mvec{G}^2 = 2\mvec{k}\cdot\mvec{G}.
\end{equation*}

Andere Form:
\begin{equation*}
	2d\sin\theta = m\lambda\quad m\in\mathbb{Z}.
\end{equation*}

\section{Die erste Brillouin-Zone}
Die erste Brillouin-Zone ist die Wigner-Seitz-Zelle des reziproken Gitters.
Nach der ersten Brillouin-Zone wiederholt sich die gesamte Kristallstruktur periodisch, d.h. es reicht, alle Prozesse in der ersten Brillouin-Zone zu beschreiben, um die Physik des Kristalles vollständig zu charakterisieren.

\section{Atomstrukturfaktoren}
Wie bereits aus der Streuung bekannt, ist die Streuamplitude in Bornscher Näherung die FT der Ladungsverteilung
\begin{equation*}
	f(\mvec{k})\propto\int_{-\infty}^\infty\text{d}^3r\rho(\mvec{r})e^{-i\Delta\mvec{k}\cdot\mvec{r}} = N\cdot\underbrace{\int_\text{EZ}\text{d}^3r\rho(\mvec{r})e^{-i\mvec{K}\cdot\mvec{r}}}_\text{Strukturfaktor}.
\end{equation*}
Befinden sich nun $s$ Atome in der Einheitszelle, so können wir den Strukturfaktor schreiben als
\begin{equation*}
	S_{\mvec{K}} = \sum_j^se^{-i\mvec{K}\cdot\mvec{r_j}}\underbrace{\int_\text{EZ}\text{d}^3r\rho(\mvec{r}-\mvec{r_j})e^{-i\mvec{K}\cdot(\mvec{r}-\mvec{r_j})}}_\text{Atomformfaktoren $f_j$}.
\end{equation*}
Ersetzen wir schlussendlich $\mvec{K}\cdot\mvec{r_j} = 2\pi(hx_{1j} + kx_{2j} + lx_{3j})$, so ergibt sich
\begin{equation*}
	S_{\mvec{K}}(h,k,l) = \sum_j^sf_je^{-2\pi\iu(hx_{1j} + kx_{2j} + lx_{3j})}.
\end{equation*}

Wir interpretieren:
Selbst wenn die Bragg-Bedingung für eine Beugung erfüllt ist, hängt es von der Basis ab, ob der Strukturfaktor 0 ist und somit kein Beugungsreflex sichtbar ist.

\textbf{Beispiel}  bcc, z.B. Metall-Na. Kubische Einheitszelle mit Basisatomen $\mvec{r_1} = (000)^\text{T}$ und $\mvec{r_2} = \frac{1}{2}(111)^\text{T}$.
\begin{equation*}
	S_{\mvec{K}}(h,k,l) = f\cdot\begin{cases}
																0,\quad(h+k+l)\text{ gerade} \\
																2,\quad(h+k+l)\text{ ungerade}
															\end{cases}
\end{equation*}
Also liefern in diesem Fall nur Miller-Ebenen mit $(h+k+l)$ ungerade einen Beugungsreflex.

\section{Der Debye-Waller-Faktor}
Betrachtet man endliche Temperaturen $T$, so sind die Gitteratome nicht stationär, sondern führen kleine Oszillationen aus.
Unter einer harmonischen Näherung dieser Oszillationen ergibt sich für die Intensität einer gestreuten Welle dann
\begin{equation*}
	I = I_0\cdot e^{-\frac{k_\text{B}T}{m\omega^2}\mvec{K}^2}.
\end{equation*}
Die Beugungsintensität nimmt also mit steigender Temperatur ab.

\section{Beugungsverfahren}
\subsection{Debye-Scherrer}
Röntgenstrahlen treffen auf kristallines Pulver, in dem die Kristalliten in zufälliger räumlicher Anordnung vorliegen.
Ein paar Kristalliten erfüllen mit ihrer Orientierung dabei die Bragg-Bedingung und werden gebeugt.
Aus den Durchmessern der Beugungsringe im Diffraktogramm lässt sich dann über die Bragg-Bedingung $2d\sin\theta = m\lambda$ Rückschluss auf den Netzebenenabstand $d$ und unter Miteinbeziehung der Millerschen Indizes dann auch die Gitterkonstanten ziehen.

\subsection{Laue}
Polychromatische Röntgenstrahlen treffen auf Einkristall.
Es sind Beugungspunktreflexe zu sehen.
Dadurch, dass die Röntgenstrahlen polychromatisch sind, erfüllen mehrere Gitterschichten die Bragg-Bedingung gleichzeitig.
Dadurch lässt sich aber ein bestimmter Reflex nicht mehr eindeutig einer Wellenlänge zuordnen.
Die Datenverarbeitung ist daher meist mühsam und zeitaufwändig.
