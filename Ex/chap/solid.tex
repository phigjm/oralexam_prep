\chapter{Festkörperphysik}

\section{Konstruktion der Wigner-Seitz-Zelle}
Einen Gitterpunkt wählen und Verbindungsstrecken zu sämtlichen anderen Gitterpunkten durch Normalebenen halbieren.
Eingegrenztes Gebiet ist dann Wigner-Seitz-Zelle.

\section{Das reziproke Gitter}
Das zu einem Ortsraumgitter reziproke Gitter ist definiert durch
\begin{equation*}
	\mvec{K} = h\mvec{A_1} + k\mvec{A_2} + l\mvec{A_3}\quad h,k,l\in\mathbb{Z}.
\end{equation*}
Dabei muss die Relation
\begin{equation*}
	\mvec{A_i}\mvec{a_j} = 2\pi\delta_{ij}
\end{equation*}
erfüllt sein.
Man kann die Vektoren $\mvec{A_i}$ durch die Konstruktionsvorschrift
\begin{equation*}
	\mvec{A_i} = 2\pi\varepsilon_{ijk}\frac{\mvec{a_k}\times\mvec{a_k}}{\mvec{a_1}\cdot(\mvec{a_2}\times\mvec{a_3})}.
\end{equation*}

\section{Beugung am Gitter}
Damit eine Beugung auftritt muss gelten
\begin{equation*}
	\Delta\mvec{k} = \mvec{K},
\end{equation*}
mit $\Delta\mvec{k}=\mvec{k}-\mvec{k'}$, dem Differenzwellenvektor der einfallenden Welle und $\mvec{K}$ einem reziproken Gittervektor.

Multiplizieren wir von links mit $\mvec{a_i}$, dann erhalten wir die \textbf{Laue-Bedingung}
\begin{align*}
	\mvec{a_1}\cdot\Delta\mvec{k} &= 2\pi h \\
	\mvec{a_2}\cdot\Delta\mvec{k} &= 2\pi k \\
	\mvec{a_3}\cdot\Delta\mvec{k} &= 2\pi l.
\end{align*}

Andere Form:
\begin{equation*}
	\mvec{G}^2 = 2\mvec{k}\cdot\mvec{G}.
\end{equation*}

Andere Form:
\begin{equation*}
	2d\sin\theta = m\lambda\quad m\in\mathbb{Z}.
\end{equation*}

\section{Die erste Brillouin-Zone}
Die erste Brillouin-Zone ist die Wigner-Seitz-Zelle des reziproken Gitters.
Nach der ersten Brillouin-Zone wiederholt sich die gesamte Kristallstruktur periodisch, d.h. es reicht, alle Prozesse in der ersten Brillouin-Zone zu beschreiben, um die Physik des Kristalles vollständig zu charakterisieren.

\section{Atomstrukturfaktoren}
Wie bereits aus der Streuung bekannt, ist die Streuamplitude in Bornscher Näherung die FT der Ladungsverteilung
\begin{equation*}
	f(\mvec{k})\propto\int_{-\infty}^\infty\text{d}^3r\rho(\mvec{r})e^{-i\Delta\mvec{k}\cdot\mvec{r}} = N\cdot\underbrace{\int_\text{EZ}\text{d}^3r\rho(\mvec{r})e^{-i\mvec{K}\cdot\mvec{r}}}_\text{Strukturfaktor}.
\end{equation*}
Befinden sich nun $s$ Atome in der Einheitszelle, so können wir den Strukturfaktor schreiben als
\begin{equation*}
	S_{\mvec{K}} = \sum_j^se^{-i\mvec{K}\cdot\mvec{r_j}}\underbrace{\int_\text{EZ}\text{d}^3r\rho(\mvec{r}-\mvec{r_j})e^{-i\mvec{K}\cdot(\mvec{r}-\mvec{r_j})}}_\text{Atomformfaktoren $f_j$}.
\end{equation*}
Ersetzen wir schlussendlich $\mvec{K}\cdot\mvec{r_j} = 2\pi(hx_{1j} + kx_{2j} + lx_{3j})$, so ergibt sich
\begin{equation*}
	S_{\mvec{K}}(h,k,l) = \sum_j^sf_je^{-2\pi\iu(hx_{1j} + kx_{2j} + lx_{3j})}.
\end{equation*}

Wir interpretieren:
Selbst wenn die Bragg-Bedingung für eine Beugung erfüllt ist, hängt es von der Basis ab, ob der Strukturfaktor 0 ist und somit kein Beugungsreflex sichtbar ist.

\textbf{Beispiel}  bcc, z.B. Metall-Na. Kubische Einheitszelle mit Basisatomen $\mvec{r_1} = (000)^\text{T}$ und $\mvec{r_2} = \frac{1}{2}(111)^\text{T}$.
\begin{equation*}
	S_{\mvec{K}}(h,k,l) = f\cdot\begin{cases}
																0,\quad(h+k+l)\text{ gerade} \\
																2,\quad(h+k+l)\text{ ungerade}
															\end{cases}
\end{equation*}
Also liefern in diesem Fall nur Miller-Ebenen mit $(h+k+l)$ ungerade einen Beugungsreflex.

\section{Der Debye-Waller-Faktor}
Betrachtet man endliche Temperaturen $T$, so sind die Gitteratome nicht stationär, sondern führen kleine Oszillationen aus.
Unter einer harmonischen Näherung dieser Oszillationen ergibt sich für die Intensität einer gestreuten Welle dann
\begin{equation*}
	I = I_0\cdot e^{-\frac{k_\text{B}T}{m\omega^2}\mvec{K}^2}.
\end{equation*}
Die Beugungsintensität nimmt also mit steigender Temperatur ab.

\section{Beugungsverfahren}
\subsection{Debye-Scherrer}
Röntgenstrahlen treffen auf kristallines Pulver, in dem die Kristalliten in zufälliger räumlicher Anordnung vorliegen.
Ein paar Kristalliten erfüllen mit ihrer Orientierung dabei die Bragg-Bedingung und werden gebeugt.
Aus den Durchmessern der Beugungsringe im Diffraktogramm lässt sich dann über die Bragg-Bedingung $2d\sin\theta = m\lambda$ Rückschluss auf den Netzebenenabstand $d$ und unter Miteinbeziehung der Millerschen Indizes dann auch die Gitterkonstanten ziehen.

\subsection{Laue}
Polychromatische Röntgenstrahlen treffen auf Einkristall.
Es sind Beugungspunktreflexe zu sehen.
Dadurch, dass die Röntgenstrahlen polychromatisch sind, erfüllen mehrere Gitterschichten die Bragg-Bedingung gleichzeitig.
Dadurch lässt sich aber ein bestimmter Reflex nicht mehr eindeutig einer Wellenlänge zuordnen.
Die Datenverarbeitung ist daher meist mühsam und zeitaufwändig.

\section{Phononen: Lineare Kette}\label{sec:linear}
Betrachten wir nun eine lineare Kette aus miteinander gekoppelten Atomen (modelliert durch Kugeln der Massen $m(M)$ verbunden durch Federn mit Federkonstante $D\rightarrow$ harmonische Näherung).

Da das System eine diskrete Natur aufweist, ist die Zahl der unterscheidbaren Vibrationsmoden endlich, wobei $\lambda_\text{min}=2a$ und $\lambda_\text{max}=2L$ (mit der Kettenlänge $L$) die minimalen und maximalen Wellenlängen der Moden sind.
Definiert man nun eine \textbf{Modenzahl} $n=\frac{2L}{\lambda_n}$, folgt die maximale Zahl der Moden
\begin{equation*}
	n_\text{max}=\frac{2L}{\lambda_\text{min}}=\frac{L}{a},
\end{equation*}
welche gleich der Anzahl Atome in der Kette ist.

Auf gleiche Weise erhalten wir die maximale Wellenzahl
\begin{equation*}
	k_\text{max}=\frac{2\pi}{\lambda_\text{min}}=\frac{\pi}{a},
\end{equation*}
die mit den Grenzen der ersten Brillouin-Zone für ein einfach kubisches Gitter übereinstimmt.

\subsection{Gleiche Massen}
\begin{figure}[tbp]
	\centering
	\includegraphics[width=0.5\textwidth]{./img/dispersion_single.pdf}
	\caption{\textbf{Dispersionsrelation für identische Atome} Jede mögliche Frequenz taucht im Intervall zwischen 0 und $\frac{\pi}{a}$ auf, was bedeutet, dass jede Wellenzahl außerhalb dieser ersten Brillouin-Zone ununterscheidbar von einer bereits darin enthaltenen Schwingung ist.}
	\label{fig:dispersion_single}
\end{figure}
Sind nur Atome von der gleichen Sorte (Masse $m$) vorhanden, lässt sich durch Aufstellen der Newtonschen Bewegungsgleichung einfach die Dispersionsrelation
\begin{equation}
	\omega(k) = \sqrt{\frac{4D}{m}}\abs{\sin\left(\frac{ka}{2}\right)}
\end{equation}
herleiten, die in \autoref{fig:dispersion_single} aufgetragen ist.

Für kleine $k$ ist $\sin(k)\approx k$, daher folgt für die Dispersionsrelation
\begin{equation*}
	\omega(k)\approx \sqrt{\frac{Da^2}{m}}\abs{k}.
\end{equation*}
Es folgt also, dass für diesen Fall, die Phasen ($v_\text{ph}=\frac{\omega}{k}$)- und Gruppengeschwindigkeit ($v_\text{gr}=\frac{\text{d}\omega}{\text{d}k}$) gleich ist
\begin{equation}\label{eq:sound_single}
	v_\text{ph} = v_\text{g}\approx \sqrt{\frac{Da^2}{m}}.
\end{equation}

\subsection{Zwei Atomsorten}
\begin{figure}[tbp]
	\centering
	\includegraphics[width=0.5\textwidth]{./img/dispersion_alternating.pdf}
	\caption{\textbf{Dispersionsrelation einer Kette mit zwei Atomsorten (abwechselnd)} Zwischen dem optischen und akustischen Zweig der Dispersionsrelation ist eine \textbf{Bandlücke} sichtbar (rot). Wellen mit Frequenzen innerhalb dieser Lücke können im Kristall nicht propagieren.}
	\label{fig:dispersion_alternating}
\end{figure}
Enthält die Kette nun zwei verschiedene Atomsorten der Massen $m_1$ und $m_2$, so hat die Dispersionsrelation nach der Lösung der Bewegungsgleichungen zwei Lösungen:
\begin{equation}
	\omega_\pm^2(k) = \frac{D}{\mu} \pm D\sqrt{\frac{1}{\mu^2} - \frac{4}{m_1 m_2}\sin^2\left(k a\right)},
\end{equation}
mit der reduzierten Masse $\mu = \frac{m_1 m_2}{m_1 + m_2}$.

Für Kristalle mit mindestens zwei verschiedenen Atomsorten in einer primtiven Elementarzelle erzeugt die Dispersionsrelation zwei Arten von Photonen, die akustischen und optischen Phononen.
Sie korrespondieren jeweils mit den $\omega_{-}$ und $\omega_{+}$ Lösungen, welche in \autoref{fig:dispersion_alternating} zu sehen sind (+ ist opt., - ist akust.).
In akustischen Moden schwingen alle Atome in Phase, während bei optischen Moden jedes Nachbaratom einen Phasenversatz von 180° relativ zum nächsten hat.
Daher entsteht eine Bandlücke in einer bestimmten Frequenzgegend, sodass Wellen mit Frequenzen innerhalb dieser Bandlücke nicht propagieren können.

Um die Dispersionsrelation von Phononen experimentell zu bestimmen, ist das beste Verfahren die inelastische Neutronenstreuung.
Neutronen wechselwirken nicht mit den Elektronen im Kristall, jedoch regen sie effizient vibronische Zustände der Atomrümpfe an.

\section{Debye-Modell}
Das Debye-Modell gibt eine Methode zur Berechnung des Beitrags der Phononen zur spezifischen Wärme eines Kristallgitters vor.
Es stellt sich heraus, dass dieser Beitrag der wesentliche ist.

Gegenüber dem Einstein-Modell, welches $N$ unabhängige Oszillatoren mit identischer Frequenz annimmt, geht das Debye-Modell von einer Vielzahl möglicher Frequenzen und einer von Null verschiedenen Ausbreitungsgeschwindigkeit aller Wellen bzw. Phononen aus.
Jedoch wird durchgehend die Langwellennäherung vorausgesetzt, das heißt, es wird der Einfachheit halber angenommen, dass bis zu einer Grenzfrequenz, der sog. Debyefrequenz $\omega_\text{D}$, immer strenge Proportionalität zwischen Frequenz und Wellenvektor (also eine lineare Dispersionsrelation) gilt,
\begin{equation*}
	\omega(k) = v_\text{s}\cdot k,
\end{equation*}
wobei ein longitudinaler und zwei transversale Schallwellen-Freiheitsgrade vorausgesetzt werden.

Die spezfische Wärme bei konstantem Volumen ist dann $\propto \left(\frac{T}{\theta_\text{D}}\right)^3$ mit der Debye-Temperatur $\theta_\text{D}=\frac{\hbar\omega_\text{max}}{k_\text{B}}$.
Die Debye-Temperatur bezeichnet dabei die Temperatur, bei der alle möglichen Zustände gerade besetzt sind.

\section{Das Bloch-Theorem}
Es sei ein periodisches Potenzial $V(\mvec{r}+\mvec{R}) = V(\mvec{r})$ gegeben.
Dann existiert eine Basis von Lösungen der stationären Schrödinger-Gleichung
\begin{equation*}
	\psi(\mvec{r}) = e^{\iu\mvec{k}\cdot\mvec{r}}\cdot u_{\mvec{k}}(\mvec{r}).
\end{equation*}
Dies sind sogenannte \textbf{Bloch-Wellen}.

Periodische Potenziale trifft man natürlich in Gittern an, weshalb dies so wichtig für die Festkörperphysik ist.

\section{Das Bänder-Modell}
Nähert man die elektronische Konfiguration innerhalb eines periodischen Gitters als frei an, so ergibt sich nach der stationären Schrödingergleichung für Elektronen eine Dispersionsrelation von
\begin{equation*}
	E(k) = \frac{\hbar^2 k_n^2}{2m},
\end{equation*}
in einer Dimension mit diskreten, erlaubten $k_n$-Werten, wie in \autoref{sec:linear} besprochen.

Die Dispersionsrelation der freien Elektronen ist also periodisch parabolisch.
Wir können die Periodizität in das reduzierte Zonenschema durch Spiegelung überführen (wenn du nicht weißt, wie das geht, dann hör' genau hier auf zu lesen und geh' deine Ex5-VL nochmal durch.).
Ist man jedoch am Rand der ersten Brillouin-Zone angekommen, so erfüllen die Elektronen gerade die Bragg-Bedingung und werden zur anderen Seite der ersten Brillouin-Zone gespiegelt.
Daher bilden die Elektronen hier stehende Wellen aus.
Die Aufenthaltswahrscheinlichkeit der Elektronen kann dabei auf zwei Arten im periodischen Potential der Kerne einrasten:
\begin{itemize}
	\item \textbf{Maxima der $|\psi|^2$ auf den Atomrümpfen:} Aufgrund der nun festeren Coulomb-Bindung der Elektronen mit den Atomrümpfen wird die Energie im Vergleich zur parabolischen Struktur dort abgesenkt (Parabel macht Knick nach unten).
	\item \textbf{Maxima der $|\psi|^2$ zwischen den Atomrümpfen:} Die Elektronen sind nun ''weiter von den Atomrümpfen entfernt'', daher wird die Energie wegen der schwächeren Coulomb-Anziehung nun angehoben (Parabel macht Knick nach oben).
\end{itemize}

Insgesamt ergibt sich durch die Parabelknicke somit eine ''Bandlücke'' von nicht erlaubten, elektronischen Zuständen.
Unter Beachtung des Pauli-Prinzips können die Bänder nun von unten mit Elektronen aufgefüllt werden, bei \SI{0}{\kelvin} bis zur Fermi-Energie $E_\text{F}$.
Bei Raumtemperatur ist die Fermi-Verteilung nur näherungsweise eine Step-Function, daher werden einige Zustände unterhalb der Fermi-Energie nicht besetzt, während einige Zustände oberhalb noch aufgefüllt werden.

Je nach Füllung der Bänder unterscheidet man zwischen:
\begin{itemize}
	\item \textbf{Isolatoren:} Valenzband voll, Leitungsband leer, große Bandlücke
	\item \textbf{Halbleitern:} Valenzband voll, Leitungsband leer, kleine Bandlücke, sodass durch Zufuhr von Energie überwunden werden kann.
	\item \textbf{Metallen:} Valenzband voll, Leitungsband nicht ganz voll, sodass Elektronen leicht durch Spannung in höhere erlaubte Niveaus innerhalb des Leitungsbandes gehoben werden können.
\end{itemize}

\section{Transport}
Es gelten die Transportgleichungen
\begin{align*}
	\grad\frac{\eta}{e} &= \frac{\mvec{j}}{\sigma} + \varepsilon\grad T \\
	\mvec{w} &= \Pi\mvec{j} - \kappa\grad T.
\end{align*}
Koeffizienten sind (wie immer) irrelevant an dieser Stelle.

Die Transportgleichungen verknüpfen somit eine elektr(ochem)ische Potenzialdifferenz $\grad\frac{\eta}{e}$ (auch: Spannung) mit einem Temperaturgradienten und vice versa.
So funktionieren Thermoelemente (basically).

\section{Bestimmung von Fermi-Oberflächen}
Die Menge der Punkte, auf die Impulsvektoren von Elektronen mit der Fermi-Energie zeigen, bilden eine geschlossene Fläche bzw. wenige geschlossene Flächen, die Fermi-Oberfläche(n) genannt wird bzw. werden.
Die Fermi-Oberfläche eines Kristalls gibt Auskunft über elektrische und magnetische Eigenschaften des Materials, da beispielsweise nur Elektronen an der Fermi-Oberfläche zum elektrischen Stromfluss beitragen.
Ferromagnetische Metalle haben daher im einfachsten Fall zwei Fermi-Flächen, je eine für die zwei möglichen Orientierungen des Spins.
Im Gegensatz dazu haben undotierte Halbleiter oder Isolatoren keine Fermi-Fläche, da deren Fermi-Energie gerade in ihrer Bandlücke liegt und es somit keine Elektronen gibt, deren Impulsvektoren auf einen erlaubten Zustand zeigen.
Bei Halbleitern verschiebt man daher durch Dotierung das Fermi-Niveau so weit, bis es in einem erlaubten Zustand liegt und somit ein Stromfluss möglich ist.

Experimentell lässt sich eine Fermi-Oberfläche z.B. durch den \textbf{De-Haas-van-Alphen-Effekt} bestimmen.

\subsection{DHvA-Effekt}
Wir stellen uns die Atome klassisch vor, d.h. das Elektron beschreibt eine elliptische Bahn um den Kern.
In minimaler Kopplung lautet der Hamiltonian des Problemes
\begin{equation*}
	H = \frac{1}{2m}\left(\mvec{p}-q\mvec{A}\right)^2.
\end{equation*}
O.B.d.A. sei das B-Feld in z-Richtung orientiert, statisch und homogen.
Dann quantisiert das System in Landau-Niveaus der Energien
\begin{equation*}
	E_n = \hbar\omega_\text{c}\left(n+\frac{1}{2}\right),
\end{equation*}
mit $\omega_\text{c}= \frac{qB}{m}$ (dabei wurde die ganze Zeit das Zeeman-Splitting außer Acht gelassen).

Ohne ein solches B-Feld hätten wir die Zustandsdichte
\begin{equation*}
	\nu(E) = \frac{2mL^2}{2\pi\hbar^2}.
\end{equation*}
Mit einem angelegten B-Feld ''kondensieren'' nun alle Zustände im Energieintervall der Breite $\hbar\omega_\text{c}$ nun auf einem Landau-Niveau, also
\begin{equation*}
	N_\text{1Niveau} = \hbar\omega_\text{c}\cdot\nu(E) = L^2\cdot\frac{2eB}{\hbar}.
\end{equation*}

Bei $n$ vollständig besetzten Landau-Levels und konstanter Elektronenzahl $N$ haben wir also in 2D (Schnittfläche Magnetfeld und Fermi-Oberfläche)
\begin{equation*}
	N = 2\frac{\pi k_\text{F}^2}{\frac{4\pi^2}{L^2}} = 2\frac{S_\text{F}}{\frac{4\pi^2}{L^2}} = n\cdot L^2\cdot\frac{2eB}{\hbar}
\end{equation*}
und daher
\begin{equation*}
	B_n^{-1} = n\cdot\frac{2\pi e}{\hbar}\cdot\frac{1}{S_\text{F}}.
\end{equation*}

Ähnliche Überlegungen erhält man für z.B. die spezifische Wärme, magnetische Suszeptibilität etc. ebenfalls Oszillationen in $\frac{1}{B}$.

Man kann aus diesen De-Haas-van-Alphen-Oszillationen nun die Periode $P$ messen und erhält über
\begin{equation*}
	P(B^{-1}) = \frac{2\pi e}{\hbar S_\text{F}}
\end{equation*}
dann die Fermi-Oberfläche.

\section{Supraleiter}
\subsection{BCS-Theorie}
Um das Phänomen der Supraleitung zu erklären, werden zwei durch das Kristallgitter fliegende Elektronen als \textbf{Cooper-Paar} betrachtet.
Das erste Elektron polarisiert das Kristallgitter auf seinem Weg geringfügig, indem es aufgrund der Coulombkraft die positiven Atomrümpfe leicht anzieht. Durch diese Verzerrung konzentriert
sich im Bahnbereich des ersten Elektrons positive Ladung.
Da die Atomrümpfe schwer und träge sind, kehren sie vergleichsweise langsam in ihre Ursprungslage zurück und bilden für das zweite Elektron noch für eine ausreichend lange Zeit eine anziehende
Bahn, entlang derer es sich energetisch günstiger bewegen kann.
Aufgrund dieser Wechselwirkung über das Kristallgitter werden die beiden Elektronen zu einem Cooper-Paar gekoppelt.
Beide Elektronen besitzen als Folge eine geringfügig abgesenkte Energie, welche die Coulombabstoßung der Elektronen überkompensieren kann, solange die beiden Elektronen selbst einen ausreichend großen Abstand besitzen.

Detaillierte Berechnungen fordern, dass Impuls und Spin beider Elektronen jeweils antiparallel sein müssen.
Da somit die Impulse der beiden gekoppelten Elektronen zu Null summieren, besitzt das Cooper-Paar keine kinetische Energie.
Damit kann es auch keine kinetische Energie durch Streuung an das Gitter abgeben, d. h. es existiert kein elektrischer Widerstand, welcher das Cooper-Paar bremst.
Die Bildung eines Cooper-Paares ist also gewissermaßen ein Austausch virtueller Phononen (durch die Gitterpolarisation).

Die Bindung der beiden Elektronen wird allerdings aufgebrochen, sobald die thermische Energie des Gitters $k_\text{B}T$ zu groß wird, also eine \textbf{Sprungtemperatur} $T_\text{c}$ überschritten wird.
Außerdem kann die Bindung aufgebrochen werden, wenn ein zu großer Strom fließt, die kinetische Energie der Elektronen die Bindung also überschreitet.

\subsection{Der Meißner-Ochsenfeld-Effekt}
Supraleiter sind unter einer Sprungtemperatur nicht nur ideale Leiter, sondern auch perfekte Diamagneten.
Sie verdrängen eine magnetische Flussdichte $B$ vollständig aus dem Material, sofern eine kritische Flussdichte $B_\text{c}$ nicht überschritten wird.
Diese Verdrängung resultiert aus Strömen, die an der Oberfläche des Materials induziert werden und daher ein Gegenfeld erzeugen.

\subsection{Klassifikation von Supraleitern}
\begin{itemize}
	\item \textbf{Supraleiter 1. Art} verdrängen das B-Feld vollständig für $B<B_\text{c}$.
	\item \textbf{Supraleiter 2. Art} verdrängen das B-Feld nur noch bei kleineren Grenzwerten $B<B_{c1}<B_c$ vollständig, gehen aber erst für größere Grenzwerte $B>B_{c2}>B_c$ in den normalleitenden Zustand über.
	Dringt das B-Feld teilweise in den Supraleiter ein, also $B_{c1}<B<B_{c2}$, so spricht man von der \textit{Shubnikow-Phase}.
	Das B-Feld dringt dabei in sogenannten \textbf{Flussschläuchen} in das Material ein.
\end{itemize}
