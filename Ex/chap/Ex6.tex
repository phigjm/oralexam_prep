\chapter{Ex 6}

\section{Teilchenstreuung}

\subsection{Mandelstam-Variablen}
Betrachten die Reaktion zweier Teilchen mit den 4-Impulsen $p_1$ und $p_2$. Die auslaufenden Teilchen sollen die 4-Impulse $p_3$ und $p_4$ haben.
Dann kann man die sogen. \textbf{Mandelstam-Variablen}
\begin{alignat*}{3}
	s &= \left(p_1 + p_2 \right)^2 &&= \left(p_3 + p_4 \right)^2 \\
	t &= \left(p_1 - p_3 \right)^2 &&= \left(p_4 - p_2 \right)^2 \\
	u &= \left(p_1 - p_4 \right)^2 &&= \left(p_3 - p_2 \right)^2.
\end{alignat*}
$s$ ist dabei das Quadrat der Schwerpunktsenergie des Systems und $u$ entspricht dem Quadrat des 4-Impulsübertrags einer gewöhnlichen Streuung.
Die Mandelstam-Variablen sind Lorentzskalare.

\subsection{Die Rapidität}
Die Rapidität ist ein Maß für Geschwindigkeiten in der Relativistik.
Sie ist definiert als
\begin{equation*}
	y = \frac{1}{2}\log\left(\frac{E+cp}{E-cp}\right)\quad \text{bzw. als}\quad y_L=\frac{1}{2}\log\left(\frac{E+cp_\text{z}}{E-cp_\text{z}}\right)
\end{equation*}
in der Teilchenphysik.
Anders als die Geschwindigkeit $v$ eines Teilchens ist die Rapidität nicht auf das Intervalll $[-c,\ +c]$ beschränkt, sondern ist unbeschränkt.
Sie gibt ein intuitiveres Gefühl dafür, welche Geschwindigkeit ein Teilchen hätte, würde es keine relativistischen Effekte erfahren.
Außerdem sind Differenzen von zwei Rapiditäten invariant unter Lorentzboosts entlang der Strahlachse und man kann Rapiditäten einfach addieren, anders als Geschwindigkeiten ($\rightarrow$ rel. Geschwindigkeitsaddition).
Aufgrund dieser Invarianz unter Lorentzboosts wird die Angabe von $y$ gegenüber dem Polarwinkel $\theta$ bevorzugt.

Für den Fall $E\gg m$ geht die Rapidität über in die sogen. \textbf{Pseudorapidität}
\begin{equation*}
	\eta = -\log\left(\tan\left(\frac{\theta}{2}\right)\right).
\end{equation*}

Der diferenzielle Wirkungsquerschnitt $\nicefrac{\text{d}\sigma}{\text{d}y}$ ist forminvariant unter Lorentzboosts entlang der Strahlachse.
Das gleiche gilt in guter Näherung auch für $\nicefrac{\text{d}\sigma}{\text{d}\eta}$, allerdings ist $\eta$ einfacher zu messen, da man nur die Flugrichtung des Teilchens durch den Detektor bestimmen muss.

\subsection{Wirkungsquerschnitte}
Streuexperimente dienen dazu, Mikroskopie über das sichtbare hinaus zu betreiben und so Erkenntnisse über die Struktur der Materie zu gewinnen.
Man unterscheidet dabei zwischen \textbf{elastischer} ($\sum E_{\text{kin,}i} = \sum E_{\text{kin,}f}$) und \textbf{inelastischer Streuung} ($\sum E_{\text{kin,}i} \neq \sum E_{\text{kin,}f}$, z.B. durch Anregung eines Atoms durch Teil der kinetischen Energie).

Der Wirkungsquerschnitt ist die wichtigste Größe bei solchen Experimenten und ist ein Maß dafür, wie wahrscheinlich eine Wechselwirkung zwischen dem Projektil und dem Target ist.
Experimentell gilt
\begin{equation*}
	\sigma = \frac{N_\text{obs}-N_\text{bg}}{\mathcal{L}\cdot\varepsilon\cdot A}\cdot\frac{1}{T},
\end{equation*}
mit der weiteren wichtigen Detektorkenngröße, der \textbf{Luminosität} $\mathcal{L}$.

\subsubsection{Rutherford-Wirkungsquerschnitt}
Der Rutherfordsche Wirkungsquerschnitt gibt den theoretischen Wirkungsquerschnitt unter Vernachlässigung von Spins und Targetrückstoß an
\begin{equation}\label{eq:ruth_sigma}
	\left(\diff[\sigma]{\Omega}\right)_\text{Ruth} = \left(\frac{zZe^2}{4\pi\epsilon_0\cdot 4E_\text{kin}}\right)^2\cdot\frac{1}{\sin^4\frac{\theta}{2}}.
\end{equation}

Definiert man den Impulsübertrag
\begin{equation*}
	\mvec{q} = \mvec{p}-\mvec{p'},
\end{equation*}
so lässt sich \autoref{eq:ruth_sigma} auch schreiben als
\begin{equation*}
	\left(\diff[\sigma]{\Omega}\right)_\text{Ruth} = \frac{4zZ\alpha^2(\hbar c)^2 E'^2}{|\mvec{q}c|^4}.
\end{equation*}
Durch die $\nicefrac{1}{q^4}$-Abhängigkeit ist es schwer, Prozesse mit hohem Impulsübertrag zu messen, da ein Fehler auf $\mvec{q}$ sich erheblich auf die Ergebnisse auswirkt.

Beachtet man, dass für kleine Massen $E\approx pc$ gilt, folgt daraus der \textbf{relativistische Rutherford-Wirkungsquerschnitt}
\begin{equation*}
	\left(\diff[\sigma]{\Omega}\right)_\text{Ruth} = \left(\frac{zZ\alpha\hbar c}{4mc^2(\gamma-1)}\right)^2\cdot\left(\frac{4p_1p_3}{q^2}\right)^2.
\end{equation*}

\subsubsection{Mott-Wirkungsquerschnitt}
Ist nun der Spin des Projektiles $\neq 0$, ergibt sich daraus der \textbf{Mott-Wirkungsquerschnitt}
\begin{equation*}
	\left(\diff[\sigma]{\Omega}\right)_\text{Mott} = \left(\diff[\sigma]{\Omega}\right)_\text{Ruth}\cdot\underbrace{\frac{E_\text{f}}{E}}_\text{Targetrückstoß}\cdot\underbrace{\left(1-\beta^2\sin^2\frac{\theta}{2}\right)}_\text{Spin d. Projektils}.
\end{equation*}
Für $\beta\rightarrow 1$ geht der Wirkungsquerschnitt über in
\begin{equation*}
	\left(\diff[\sigma]{\Omega}\right)_\text{Mott}= \rightarrow\left(\diff[\sigma]{\Omega}\right)_\text{Ruth}\cdot\frac{E_\text{f}}{E}\cdot\cos^2\frac{\theta}{2}.
\end{equation*}
Dabei sieht man, dass für $\beta\rightarrow 1$ ein Streuwinkel von 180° nicht möglich ist.
Dies folgt aus der Helizitätserhaltung: Bei einer Rückstreuung müsste auch der Spin flippen, damit die Projektion des Spins auf die Flugbahn (Helizität) erhalten bleibt. Dies ist jedoch an einem spinlosen Target aufgrund der Drehimpulserhaltung nicht möglich.

\subsubsection{Dirac-Wirkungsquerschnitt}
Bei der \textbf{Dirac-Streuung} haben nun beide Streupartner einen Spin.
Der Wirkungsquerschnitt wird dementsprechend modifiziert
\begin{equation*}
	\left(\diff[\sigma]{\Omega}\right)_\text{Dirac} = \left(\diff[\sigma]{\Omega}\right)_\text{Mott}\cdot\left(1+2\tau\tan^2\frac{\theta}{2}\right),
\end{equation*}
wobei $\tau = \frac{Q^2}{4M^2c^2}$.
Damit kann allerdings keine Streuung z.B. an realen Protonen beschrieben werden, da hier angenommen wird, dass das \textbf{punktförmige} ''Dirac-Proton'' ein herkömmliches magnetisches Moment von $\mu=\frac{e}{2m}$ besitzt.
