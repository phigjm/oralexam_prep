\chapter{Ex 6}

\section{Teilchenstreuung}

\subsection{Mandelstam-Variablen}
Betrachten die Reaktion zweier Teilchen mit den 4-Impulsen $p_1$ und $p_2$. Die auslaufenden Teilchen sollen die 4-Impulse $p_3$ und $p_4$ haben.
Dann kann man die sogen. \textbf{Mandelstam-Variablen}
\begin{alignat*}{3}
	s &= \left(p_1 + p_2 \right)^2 &&= \left(p_3 + p_4 \right)^2 \\
	t &= \left(p_1 - p_3 \right)^2 &&= \left(p_4 - p_2 \right)^2 \\
	u &= \left(p_1 - p_4 \right)^2 &&= \left(p_3 - p_2 \right)^2.
\end{alignat*}
$s$ ist dabei das Quadrat der Schwerpunktsenergie des Systems und $u$ entspricht dem Quadrat des 4-Impulsübertrags einer gewöhnlichen Streuung.
Die Mandelstam-Variablen sind Lorentzskalare.

\subsection{Die Rapidität}
Die Rapidität ist ein Maß für Geschwindigkeiten in der Relativistik.
Sie ist definiert als
\begin{equation*}
	y = \frac{1}{2}\log\left(\frac{E+cp}{E-cp}\right)\quad \text{bzw. als}\quad y_L=\frac{1}{2}\log\left(\frac{E+cp_\text{z}}{E-cp_\text{z}}\right)
\end{equation*}
in der Teilchenphysik.
Anders als die Geschwindigkeit $v$ eines Teilchens ist die Rapidität nicht auf das Intervalll $[-c,\ +c]$ beschränkt, sondern ist unbeschränkt.
Sie gibt ein intuitiveres Gefühl dafür, welche Geschwindigkeit ein Teilchen hätte, würde es keine relativistischen Effekte erfahren.
Außerdem sind Differenzen von zwei Rapiditäten invariant unter Lorentzboosts entlang der Strahlachse und man kann Rapiditäten einfach addieren, anders als Geschwindigkeiten ($\rightarrow$ rel. Geschwindigkeitsaddition).
Aufgrund dieser Invarianz unter Lorentzboosts wird die Angabe von $y$ gegenüber dem Polarwinkel $\theta$ bevorzugt.

Für den Fall $E\gg m$ geht die Rapidität über in die sogen. \textbf{Pseudorapidität}
\begin{equation*}
	\eta = -\log\left(\tan\left(\frac{\theta}{2}\right)\right).
\end{equation*}

Der diferenzielle Wirkungsquerschnitt $\nicefrac{\text{d}\sigma}{\text{d}y}$ ist forminvariant unter Lorentzboosts entlang der Strahlachse.
Das gleiche gilt in guter Näherung auch für $\nicefrac{\text{d}\sigma}{\text{d}\eta}$, allerdings ist $\eta$ einfacher zu messen, da man nur die Flugrichtung des Teilchens durch den Detektor bestimmen muss.
