\chapter{Tiefinelastische Streuung}
Um die Substruktur zusammengesetzter Teilchen zu erforschen, braucht man eine gute Auflösung.
Dabei muss die Wellenlänge der ausgetauschten, virtuellen Photonen viel kleiner als die räumliche Ausdehnung des Nukelons sein
\begin{equation*}
	\frac{\lambda}{2\pi}\ll R\quad\Rightarrow\quad Q^2 \gg \left(\frac{\hbar}{R}\right)^2.
\end{equation*}
Derartig große Impulsüberträge fordern hohe Strahlenergien.
Die ersten Experimente dieser Art wurden in den 70ern am SLAC durchgeführt, einem Linearbeschleuniger für Elektronen mit ca. \SI{25}{\GeV}.
Bei der zweiten Generation dieser Experimente, welche in den 80ern am CERN und FNAL stattfanden, wurden Myonen anstelle von Elektronen verwendet.
Diese verhalten sich in jeder Hinsicht wie Elektronen, da sie geladene, punktförmige Teilchen mit Spin-$\nicefrac{1}{2}$ sind.
Allerdings bieten Myonen den Vorteil, mit größeren Energien erzeugt werden zu können.
Man beschießt dazu Targets mit hochenergetischen Protonen, wobei viele Pionen entstehen.
Ein Teil dieser Pionen zerfällt auf dem Flug in Myonen.
Diese Werden nach Impulsen selektiert und mit Magnetlinsen zu einem Strahl gebündelt.
So ließen sich am CERN Strahlenergien von bis zu \SI{280}{\GeV} erzeugen, am FNAL sogar \SI{490}{\GeV}.
Die letzte Generation wurde am HERA-Speicherring realisiert.
Dort zirkulieren Elektronen bzw. Positronen mit einer Energie von \SI{27.6}{\GeV} und Protonen mit einer maximalen Energie von \SI{920}{\GeV} in separaten Speicherringen und werden an einer Stelle zur Kollision gebracht.
Damit lassen sich Strahlenergien im fünfstelligen \si{\GeV}-Bereich erreichen.

\section{Tiefinelastische Anregungen}
\begin{figure}
	\centering
	\includegraphics{./img/resonance_deep.pdf}
	\caption{\textbf{Spektrum einer Elektron-Proton-Streuung}}
	\label{fig:resonance}
\end{figure}
\autoref{fig:resonance} zeigt ein Elektron-Proton-Streuspektrum bei einer Elektronenergie von \SI{4.9}{\GeV} und einem Streuwinkel von \SI{10}{\degree}.
Neben einem scharfen Maximum, das von der elastischen Streuung herrührt, sind weitere kleinere Maxima zu sehen.
Diese resultieren aus inelastischen Anregungen des Protons.
Man nennt diese angeregten Nukleonzustände auch \textbf{Nukleonresonanzen}.
Die Tatsache, dass solche angeregten Zustände existieren, ist bereits ein Indiz dafür, dass das Nukleon ein zusammengesetztes System ist.

\section{Strukturfunktionen}
\begin{figure}
	\centering
	\includegraphics[width=.5\textwidth]{./img/formfactor_deep.pdf}
	\caption{\textbf{Strukturfunktion $F_2$ des Protons als Funktion von $x$} bei verschiedenen $Q^2$-Werten im angegebenen Bereich}
	\label{fig:formfactor_deep}
\end{figure}
Bei invarianten Massen $W = \left|\frac{p'}{c}\right|$ die etwa größer als \SI{2.5}{\GeV} sind, sieht man im Spektrum keine scharfen Resonanzen mehr, aber beobachtet eine vermehrte Erzeugung von Hadronen.
Diese Prozesse heißen \textbf{tiefinelastische Streuung}.
Man kann analog zur Rosenbluth-Formel den doppelten Wirkungsquerschnitt
\begin{equation*}
	\ddiff[\sigma]{\Omega}{E'} = \left(\diff[\sigma]{\Omega}\right)_\text{Mott}\cdot\left(W_2(Q^2,\nu) + 2W_1(Q^2,\nu)\cdot\tan^2\frac{\theta}{2}\right)
\end{equation*}
definieren.
Diese Strukturfunktionen haben nun aufgrund der Anregungsenergie des Protons \textbf{zwei} freie Parameter.
Anstelle der dimensionsbehafteten Funktionen $W_i$ benutzt man heutzutage nur noch die dimensionslosen Strukturfunktionen
\begin{align*}
	F_1(x,Q^2) &= Mc^2W_1(Q^2,\nu) \\
	F_2(x,Q^2) &= \nu W_2(Q^2,\nu)
\end{align*}
mit den beiden Lorentzinvarianten Variablen $Q^2$ und $x=\frac{Q^2}{2Pq}=\frac{Q^2}{2M\nu}$.
Man nennt $x$ auch die \textbf{Bjorkensche Skalenvariable}.
Mit diesen neuen Größen lässt sich der Wirkungsquerschnitt lorentzinvariant
\begin{equation*}
	\ddiff[\sigma]{Q^2}{x} = \frac{4\pi\alpha^2\hbar^2}{Q^4}\left(\left(\frac{1-y}{x}-\frac{My}{2E}\right)F_2(Q^2,x) + y^2F_1(Q^2,x)\right)
\end{equation*}
schreiben.
Dabei ist $y = \frac{Pq}{Pp}\stackrel{\text{Lab}}{=} 1-\frac{E'}{E}$, $P$ der Target-4-Impuls und $p$ der Projektil-4-Impuls.

\autoref{fig:formfactor_deep} zeigt eine Messung der Strukturfunktion $F_2(x)$ bei verschiedenen $Q^2$-Werten im Bereich von \SIrange{2}{18}{\GeV}.
Erstaunlicherweise bemerkt man, dass die Funktion eine kleine Abhängigkeit von $Q^2$ aufweist.
Dies bedeutet allerdings, dass man an punktförmigen Teilchen gestreut hat ($\rightarrow$ Diskussion Rosenbluth-Plot).
Also kann man die Folgerung machen, dass das Nukleon aus punktförmigen Konstituenten aufgebaut ist.
Nimmt man an, dass für die Masse dieser Konstituenten $m_x = x\cdot m_p$ gilt und dass sie einen Spin von $\nicefrac{1}{2}$ haben, so folgt die \textbf{Callan-Gross-Beziehung}
\begin{equation*}
	2xF_1(x) = F_2(x).
\end{equation*}
Messungen bestätigen diese Beziehung, womit auch der Spin dieser sogenannten \textbf{Partonen} geklärt ist.
