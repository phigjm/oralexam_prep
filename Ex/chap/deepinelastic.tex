\chapter{Tiefinelastische Streuung}
Um die Substruktur zusammengesetzter Teilchen zu erforschen, braucht man eine gute Auflösung.
Dabei muss die Wellenlänge der ausgetauschten, virtuellen Photonen viel kleiner als die räumliche Ausdehnung des Nukelons sein
\begin{equation*}
	\frac{\lambda}{2\pi}\ll R\quad\Rightarrow\quad Q^2 \gg \left(\frac{\hbar}{R}\right)^2.
\end{equation*}
Derartig große Impulsüberträge fordern hohe Strahlenergien.
Die ersten Experimente dieser Art wurden in den 70ern am SLAC durchgeführt, einem Linearbeschleuniger für Elektronen mit ca. \SI{25}{\GeV}.
Bei der zweiten Generation dieser Experimente, welche in den 80ern am CERN und FNAL stattfanden, wurden Myonen anstelle von Elektronen verwendet.
Diese verhalten sich in jeder Hinsicht wie Elektronen, da sie geladene, punktförmige Teilchen mit Spin-$\nicefrac{1}{2}$ sind.
Allerdings bieten Myonen den Vorteil, mit größeren Energien erzeugt werden zu können.
Man beschießt dazu Targets mit hochenergetischen Protonen, wobei viele Pionen entstehen.
Ein Teil dieser Pionen zerfällt auf dem Flug in Myonen.
Diese Werden nach Impulsen selektiert und mit Magnetlinsen zu einem Strahl gebündelt.
So ließen sich am CERN Strahlenergien von bis zu \SI{280}{\GeV} erzeugen, am FNAL sogar \SI{490}{\GeV}.
Die letzte Generation wurde am HERA-Speicherring realisiert.
Dort zirkulieren Elektronen bzw. Positronen mit einer Energie von \SI{27.6}{\GeV} und Protonen mit einer maximalen Energie von \SI{920}{\GeV} in separaten Speicherringen und werden an einer Stelle zur Kollision gebracht.
Damit lassen sich Strahlenergien im fünfstelligen \si{\GeV}-Bereich erreichen.
