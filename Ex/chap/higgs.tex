\chapter{Higgs}
So elegant die Theorie der elektroschwachen Vereinigung auch ist, hat sie einen gravierenden Schönheitsfehler.
Man betrachte den naiven Massenterm in der Lagrangedichte des SM unter einer Eichtransformation $A_\mu \rightarrow A'_\mu = A_\mu + \frac{1}{e}\partial_\mu\theta$
\begin{equation*}
	\mathcal{L}_M \propto m_A A_\mu A^{\mu*}\longrightarrow m_A A'_\mu A^{'\mu*} = m_A A_\mu A^{\mu*} + \underbrace{\frac{m_A}{e}(A_\mu\partial^\mu\theta + A^{\mu*}\partial_\mu\theta) + \frac{m_A}{e^2}\partial_\mu\theta\partial^\mu\theta}_{\neq 0}.
\end{equation*}
Wie man sehen kann, ist die Lagrangedichte für massive Eichbosonen nicht eichinvariant!

Man könnte vermuten, dass der nicht-abelsche Charakter der $SU(2)_L$ der Grund für diese Eichinvarianz ist, dem ist aber nicht so. (\textbf{Warum?}) %TODO
Es liegt vielmehr an der Unterscheidung zwischen links- und rechtshändigen Teilchen.
Um dieses Problem zu lösen, behilft man sich einem Konzept aus der Phyik der Phasenübergänge und postuliert einen neuen Sektor, der eine \textit{spontane Symmetriebrechung} erfährt.
In diesem Modell werden sogenannte \textit{Higgs-Felder} postuliert.
Bei ausreichend hohen Temperaturen bzw. Energien sind $Z^0$- und $W^\pm$-Bosonen sowie das Photon masselos.
Unterhalb der Energie des Phasenüberganges werden die Massen der Bosonen dann durch die Kopplung an die Higgs-Felder erzeugt.
In der Theorie der elektroschwachen Vereinheitlichung gibt es 4 Higgs-Felder.
Bei kleinen Energien werden 3 Higgs-Bosonen, die Quanten des Higgs-Feldes von den elektroschwachen Bosonen absorbiert und erhalten somit ihre Masse.
Das Photon bleibt weiterhin masselos, daher muss es ein freies, massives Higgs-Boson geben. 
