\chapter{Quarks, Gluonen und die starke Wechselwirkung}

\section{Quarks in Hadronen}
Neben den Nukleonen gibt es noch viele instabile Hadronen.
Man teilt Hadronen in zwei Klassen auf:
\begin{itemize}
	\item \textbf{Baryonen} \begin{itemize}
		\item halbzahliger Spin, daher Fermionen
		\item Protonen, Neutronen\dots
	\end{itemize}
	\item \textbf{Mesonen} \begin{itemize}
		\item ganzzahliger Spin, daher Bosonen
		\item Pionen, Kaonen,\dots
	\end{itemize}
\end{itemize}

\subsection{Baryonen}
Alle Baryonen sind aus \textbf{3 Quarks} zusammengesetzt.
Da die Quarks Spin $\tfrac{1}{2}$ haben, ergibt sich daraus der halbzahlige Baryonen-Spin.
Wenn in Teilchenreaktionen zusätzliche Baryonen erzeugt werden, dann wird zugleich die gleiche Zahl an Antibaryonen erzeugt.
Zur Beschreibung führt man die additive \textbf{Baryonenzahl $B$} ein.
Sie beträgt 1 für alle Baryonen und -1 für alle Antibaryonen.
(Anti)Quarks haben demnach die Baryonenzahl ($-$)$\tfrac{1}{3}$.

\textbf{Die Baryonenzahl ist in jeden bekannten Teilchenreaktionen eine Erhaltungsgröße.}

\subsection{Mesonen}
Hadronen aus Quark-Antiquark-Paaren nennt man \textbf{Mesonen}.
Ihr ganzzahliger Spin setzt sich aus der Spinkopplung des Quarkpaares und etwaigen ganzzahligen Bahndrehimpulsen zusammen.
Es gibt keine Mesonenzahlerhaltung.
Dies ergibt Sinn, da Mesonen Quark-Antiquark-Kombinationen sind (ihre Baryonenzahl somit 0 ist), wodurch beliebig viele Mesonen erzeugt werden können ohne die Baryonenzahlerhaltung zu verletzen.

\section{Quark-Gluon-Wechselwirkung}
\subsection{Die Farbe}
Die \textbf{Farbe} wird benötigt, um das Pauli-Prinzip für die Quarks der Hadronen zu gewährleisten.

Betrachten wir beispielsweise die $\Delta^{++}$-Resonanz.
Der Spin dieses Baryons beträgt $J=\nicefrac{3}{2}$ und die Parität ist positiv.
Da das $\Delta^{++}$ das leichteste Baryon mit $J^P=\nicefrac{3}{2}^+$ ist, kann man annehmen, dass der Bahndrehimpuls verschwindet, die Ortswellenfunktion somit symmetrisch ist.
Damit sich der Spin von $\tfrac{3}{2}$ ergibt, müssen alle Spins parallel sein.
Damit ist auch die Spinwellenfunktion symmetrisch.
Ferner ist die Wellenfunktion auch unter Vertauschung zweier Quarks symmetrisch, da nur Quarks derselben Sorte vorhanden sind.
Damit scheint die Gesamtwellenfunktion symmetrisch zu sein, was aber gegen das Pauli-Prinzip verstößt.

Es lässt sich allerdings dadurch retten, indem man eine neue Quantenzahl einführt: Die \textbf{Farbe}.
Sie kann drei Werte (rot, blau, grün) annehmen.
Entsprechend gibt es für Antiquarks auch Farben (antirot, antiblau, antigrün).
Hiermit kann man eine unter Quarkvertauschung antisymmetrische Farbwellenfunktion konstruieren, sodass die Gesamtwellenfunktion antisymmetrisch ist
