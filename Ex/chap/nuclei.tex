\chapter{Kerne}

\section{Kernmodelle}
Die Masse eines Kernes ist
\begin{equation*}
	m_\text{Kern} = Z\cdot m_\text{p} + N\cdot m_\text{n} - E_\text{B}
\end{equation*}
mit $Z$, der Kernladungszahl, $N$, der Neutronenzahl und $E_\text{B}$, der Bindungsenergie.

Es gibt verschiedene Modelle, diese Bindungsenergie zu beschreiben.

\subsection{Tröpfchenmodell}
Das Tröpfchenmodell beschreibt einen Atomkern wie einen Flüssigkeitstropfen.
Die darauf beruhende \textbf{Bethe-Weizsäcker-Massenformel} lautet
\begin{equation*}
	E_\text{B} = \underbrace{a_\text{V}A}_{E_\text{V}} - \underbrace{a_\text{S}A^\nicefrac{2}{3}}_{E_\text{S}} - \underbrace{a_\text{C}\frac{Z(Z-1)}{A^\nicefrac{1}{3}}}_{E_\text{C}} - \underbrace{a_\text{A}\frac{(A-2Z)^2}{A}}_{E_\text{C}} + \delta (A,Z).
\end{equation*}

\subsubsection{Der Volumenterm $E_\text{V}$}
Der Volumenterm $E_\text{V}$ ist proportional zum Volumen $V$ des Kerns, welcher proportional zu $A$ ist.
Es liegt dabei die starke Wechselwirkung zugrunde, die bekanntlich gleichermaßen auf Protonen und Neutronen wirkt, daher überrascht es nicht, dass kein $Z$ im Term vorkommt.
Die Zahl der möglichen Wechselwirkungspaare ist $\frac{A(A-1)}{2}$, also würde man naiv einen Term erwarten, der wie $A^2$ geht.
Allerdings hat die starke Wechselwirkung nur eine sehr begrenzte Reichweite, sodass nur nächste und übernächste Nachbarn wechselwirken können, daher geht der Term etwa linear in $A$.

\subsubsection{Der Oberflächenterm $E_\text{S}$}
Der Oberflächenterm $E_\text{S}$ basiert ebenfalls auf der starken Wechselwirkung.
Er ist eine Korrektur zum Volumenterm für Teilchen am Rand des Tropfens, der beachtet, dass Nukleonen an der Oberfläche des Tropfens weniger Wechselwirkungspartner zur Verfügung haben.
Wenn also das Volumen des Tropfens $\propto A$ ist, dann ist die Oberfläche $\propto A^\nicefrac{2}{3}$, daher die Form.
Das Analogon zum Tropfen ist hier die Oberflächenspannung.

\subsubsection{Der Coulombterm $E_\text{C}$}
Dem Coulombterm $E_\text{C}$ liegt die elektrostatische Abstoßung zwischen den Protonen zugrunde.
Da die Abstoßung nur für mehr als ein Proton existieren kann, folgt die Form $Z(Z-1)$.
Wenn man annimmt, dass der Tropfen eine gleichmäßige Ladungsverteilung hat, so geht das elektrische Feld mit $\frac{1}{R}\propto\frac{1}{A^\nicefrac{1}{3}}$.

\subsubsection{Der Asymmetrieterm $E_\text{A}$}
Beim Asymmetrieterm $E_\text{A}$ kommt das Pauli-Prinzip zum Tragen.
Sowohl Protonen, als auch Neutronen können unter Beachtung des Pauli-Prinzips innerhalb ihrer eigenen Pools Energieniveaus bis zu ihrer Fermienergie auffüllen.
Wenn nun von einer Sorte Nukleonen mehr Teilchen im Kern vorhanden sind, so hat dieser Pool eine höhere Fermienergie als der andere.
Wir könnten durch einen schwachen Zerfall nun die Energie senken, daher ist die Energie höher, als sie sein müsste.
Diese Überlegung bildet die Basis dieses Termes.

Im Fermigasmodell für Kerne folgt dieser Term sogar explizit aus der Rechnung.

\subsection{Das Fermigasmodell}
