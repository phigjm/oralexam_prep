\chapter{Atome und Moleküle}

\section{Die Plankverteilung}
Im Rahmen der klassischen Physik kann man die spektrale Leistungsverteilung eines schwarzen Körpers relativ einfach zu
\begin{equation*}
	M(\lambda)\cdot\text{d}\lambda = \frac{2\pi\cdot c\cdot k_\text{B}T}{\lambda^4}\cdot\text{d}\lambda.
\end{equation*}
Richtig ist dieses Verhalten bei großen Wellenlängen.
Jedoch strebt die Strahlungsleistung für kleine Wellenlängen gegen unendlich, was man als \textit{Ultraviolett-Katastrophe} bezeichnet.

Planck löste dieses Problem durch die Annahme, dass sich Wellen in einem Hohlraum (schwarzer Körper) sich verhalten wie harmonische Oszillatoren, die nur diskrete Energiewerte $E=n\cdot\hbar\omega$ annehmen können und bei diesen Energien Strahlung absorbieren/emittieren.
Einstein konnte die Planksche Strahlungsformel dann relativ einfach durch Mastergleichungen für das System (?) herleiten.
Die Übergänge lauteten
\begin{itemize}
	\item \textbf{Induzierte Emission:} $\text{d}N_{21} = B_{21}u(\nu)N_2\text{d}t$
	\item \textbf{Spontane Emission:} $\text{d}N_{21} = A_{21}N_2\text{d}t$.
\end{itemize}
Im thermischen Gleichgewicht folgt dann $\text{d}N_{21} = \text{d}N_{12}$ oder
\begin{equation*}
	\frac{N_2}{N_1} = \frac{B_{12}u(\nu)}{A_{21}+B_{21}u(\nu)} \stackrel{\text{Boltzm.}}{=} \frac{\exp(-E_1)}{\exp(-E_2)}.
\end{equation*}
Mit den Randbedingungen $T\rightarrow\infty : u(\nu)\rightarrow\infty$ und dass bei kleinen Frequenzen das Rayleigh-Jeans-Gesetz gelten muss, folgt dann die \textbf{spektrale Energiedichte der Photonen}
\begin{equation*}
	u(\nu) = \frac{8\pi h\nu^3}{c^3}\frac{1}{e^{\frac{h\nu}{k_\text{B}T}} - 1}.
\end{equation*}

Differenziert man diese Verteilung, ergibt sich das \textbf{Wien'sche Verschiebungsgesetz}
\begin{equation*}
	\lambda_\text{max}\cdot T = \text{const} = \SI{2.9e-3}{\kelvin\meter}.
\end{equation*}
Dieses Gesetz beschreibt die Lage des Peaks in der spektralen Leistungsverteilung der Strahlung eines Körpers.
