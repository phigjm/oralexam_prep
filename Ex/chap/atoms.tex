\chapter{Atome und Moleküle}

\section{Die Plankverteilung}
Im Rahmen der klassischen Physik kann man die spektrale Leistungsverteilung eines schwarzen Körpers relativ einfach zu
\begin{equation*}
	M(\lambda)\cdot\text{d}\lambda = \frac{2\pi\cdot c\cdot k_\text{B}T}{\lambda^4}\cdot\text{d}\lambda.
\end{equation*}
Richtig ist dieses Verhalten bei großen Wellenlängen.
Jedoch strebt die Strahlungsleistung für kleine Wellenlängen gegen unendlich, was man als \textit{Ultraviolett-Katastrophe} bezeichnet.

Planck löste dieses Problem durch die Annahme, dass sich Wellen in einem Hohlraum (schwarzer Körper) sich verhalten wie harmonische Oszillatoren, die nur diskrete Energiewerte $E=n\cdot\hbar\omega$ annehmen können und bei diesen Energien Strahlung absorbieren/emittieren.
Einstein konnte die Planksche Strahlungsformel dann relativ einfach durch Mastergleichungen für das System (?) herleiten.
Die Übergänge lauteten
\begin{itemize}
	\item \textbf{Induzierte Emission:} $\text{d}N_{21} = B_{21}u(\nu)N_2\text{d}t$
	\item \textbf{Spontane Emission:} $\text{d}N_{21} = A_{21}N_2\text{d}t$.
\end{itemize}
Im thermischen Gleichgewicht folgt dann $\text{d}N_{21} = \text{d}N_{12}$ oder
\begin{equation*}
	\frac{N_2}{N_1} = \frac{B_{12}u(\nu)}{A_{21}+B_{21}u(\nu)} \stackrel{\text{Boltzm.}}{=} \frac{\exp(-E_1)}{\exp(-E_2)}.
\end{equation*}
Mit den Randbedingungen $T\rightarrow\infty : u(\nu)\rightarrow\infty$ und dass bei kleinen Frequenzen das Rayleigh-Jeans-Gesetz gelten muss, folgt dann die \textbf{spektrale Energiedichte der Photonen}
\begin{equation*}
	u(\nu) = \frac{8\pi h\nu^3}{c^3}\frac{1}{e^{\frac{h\nu}{k_\text{B}T}} - 1}.
\end{equation*}

Differenziert man diese Verteilung, ergibt sich das \textbf{Wien'sche Verschiebungsgesetz}
\begin{equation*}
	\lambda_\text{max}\cdot T = \text{const} = \SI{2.9e-3}{\kelvin\meter}.
\end{equation*}
Dieses Gesetz beschreibt die Lage des Peaks in der spektralen Leistungsverteilung der Strahlung eines Körpers.

\section{Das H-Atom}
Ohne relativistische Korrekturen, Fein-/Hyperfreinstruktur kann man die Schrödingergleichung für das Wasserstoffatom
\begin{equation*}
	\left(\frac{\hbar^2}{2m}-V_\text{c}(r)\right)\psi(r,\theta,\varphi) = E\psi(r,\theta,\varphi)
\end{equation*}
mit $V_\text{c}(r)=-\frac{e^2}{4\pi\varepsilon_0r}$ durch einen Separationsansatz
\begin{equation*}
	\psi(r,\theta,\varphi) = R_{nl}(r)\cdot Y_{lm}(\theta,\varphi)
\end{equation*}
lösen.
Dabei sind die $R_{nl}(r)$ im Wesentlichen die zugeordneten Laguerre-Polynome und $Y_{lm}(\theta,\varphi)$ die Kugelflächenfunktionen.
Diese Lösung führt auf die unkorrigierten Eigenenergien
\begin{equation*}
	E_n = -\frac{E_\text{R}}{n^2},
\end{equation*}
also ist das System hochgradig entartet bezüglich $l,m$.
$E_\text{R}$ ist die Rydberg-Energie, Wert irrelevant.

Die Energieniveaus spalten auf, wenn man die Fein-/Hyperfreinstruktur miteinbezieht.

\section{Stern-Gerlach-Experiment}
\textbf{Anordnung}  Ein Strahl von Silberatomen durchfliegt im Vakuum den Spalt zwischen den Polschuhen eines Magneten.
Durch die spezielle Form des Magneten weist das B-Feld quer zum Strahl eine starke Inhomogenität auf.
Nachdem der Strahl das Feld durchlaufen hat, schlagen sich die Silberatome auf einer Glasplatte nieder.

\textbf{Resultat}  Es werden zwei voneinander getrennte Flecke gefunden, das heißt, das Magnetfeld spaltet den Strahl in zwei getrennte Strahlen auf.

\textbf{Erklärung}  Da sich im Silberatom in der äußersten Valenzschale (5s) ein einzelnes Elektron befindet und sich der Gesamtdrehimpuls der inneren Schalen zu 0 addiert (abgeschlossene Schalen), besteht der Gesamtdrehimpuls der Atome also nur aus dem Spin dieses einen Elektrons.
Der entscheidende Unterschied zum Elektron ist hierbei aber, dass das Silberatom elektrisch neutral ist, somit im Magnetfeld keine Lorentzkraft erfahren kann.
Ferner könnte es auch nicht durch elektrische Störfelder abgelenkt werden.

Das Silberatom hat also ein magnetisches Dipolmoment $\mvec{\mu}$, auf das im inhomogenen Feld $\mvec{B}$ eine Kraft wirkt.
Klassisch müsste sich also, je nach Anstellwinkel zur Feldrichtung, eine kontinuierliche Aufweitung des Strahles in $\pm z$-Richtung beobachten lassen.

Da aber das magnetische Dipolmoment vom Drehimpuls $\mvec{S}$ her rührt und der Spin des Elektrons sich im Feld entweder parallel oder antiparallel anordnet (Quantenzahl $m=\pm\frac{1}{2}$), beobachtet man zwei diskrete Flecken.

Je nach dem, in welchem Orbital sich ein Valenzelektron befinden würde (s,p,d,f,\dots), würde man eine Aufspaltung in ggf. mehr als nur zwei Strahlen beobachten, je nach erlaubten Quantenzahlen $m$.

\section{Spin-Bahn-Kopplung und der Paschen-Back-Effekt}
Die Spin-Bahn-Kopplung ist in einem semiklassischen Modell eine Art ''interner Zeeman-Effekt''.
Das Feld wird dabei durch die Bahnbewegung der Elektronen erzeugt (Biot-Savart-Gesetz).
Das magnetische Moment des Spins koppelt somit an den Bahndrehimpuls.
Wenn wir also diese Korrektur in den Hamiltonian miteinbeziehen wird die Entartung z.B. im H-Atom teilweise aufgehoben
\begin{equation*}
	E_{nls} \propto E_n + \Delta E_\text{LS}(j,l).
\end{equation*}
Gute Quantenzahlen sind nun $n, l, s, j, m_j$.

Zustände werden nun mit der Nomenklatur $n^{2s+1}l_j$ bezeichnet.

Wird nun allerdings ein großes, externes Magnetfeld angelegt, entkoppeln $\mvec{L}$ und $\mvec{S}$ und präzedieren unabhängig um die Richtung des magnetischen Feldes.
$J$ ist somit keine Quantenzahl von Eigenzuständen mehr.
Da $\mvec{L}$ und $\mvec{S}$ nun unabhängig an das magnetische Feld koppeln, ergibt sich ein neues Spektrum, das meistens einfacher aussieht, da weniger Zustände aufspalten.

Man nennt dieses Phänomen \textbf{Paschen-Back-Effekt}.

\section{LS-Kooplung bei Mehrelektronensystemen}
\subsection{Mehr Elektronen}
Bei moderat vielen Elektronen (bis etwa Kohlenstoff) addieren die Einzelspins und Einzelbahndrehimpulse zu Gesamtspins/-drehimpulsen
\begin{equation*}
	\mvec{L} = \sum_i\mvec{L_i}\quad\mvec{S} = \sum_i\mvec{S_i}.
\end{equation*}
Diese Gesamtdrehimpulse koppeln dann zum Gesamtdrehimpuls
\begin{equation*}
	\mvec{J} = \mvec{L} + \mvec{S}.
\end{equation*}

\subsection{Sehr viele Elektronen, hohe Kernladungszahlen}
Bei hohen Kernladungszahlen wird die Spin-Bahn-Wechselwirkung groß, weil $V_\text{LS}\propto Z^4$.
Dann liegt eine jj-Kopplung vor, bei denen für jedes Elektron für sich Spin-Bahn-Kopplung gilt
\begin{equation*}
	\mvec{j_i} = \mvec{l_i} + \mvec{s_i},
\end{equation*}
welche dann zu einem Gesamtdrehimpuls koppeln
\begin{equation*}
 \mvec{J} = \sum_i\mvec{j_i}.
\end{equation*}
