\chapter{Elektron-Positron-Kollisionen}
\section{Allgemeines}
In der Elektron-Positron-Annihilation können alle schwach und elektromagnetisch Wechselwirkenden Teilchen erzeugt werden.
Austauschbosonen sind in diesem Fall entweder das Photon $\gamma$ oder das $Z^0$-Boson.
Da Neutrinos elektrisch ungeladen sind, können Neutrino-Antineutrino-Paare nur durch den Austausch eines $Z^0$ erzeugt werden.

In einem Speicherring mit kollidierenden Teilchen gleicher Energie beträgt die maximale Schwerpunktsenergie $\sqrt{s}=2E$, womit alle Teilchen-Antiteilchen-Paare der Massen $2m = \nicefrac{\sqrt{s}}{c^2}$ erzeugt werden können.
Anders als bei Fixed-Target-Experimenten steht in Collider-Experimenten die gesamte Schwerpunktsenergie zur Verfügung, während im anderen Fall nur $s\approx 2MEc^2$ genutzt werden können, die Schwerpunktsenergie wächst also nur mit der Wurzel der Strahlenergie.

\section{Erzeugung von Leptonenpaaren}
\subsection{Myonen}
Sie sind die leichtesten Teilchen, die in $\el\pos$-Reaktionen erzeugt werden können:
\begin{equation*}
	\el + \pos \rightarrow \mu^- + \mu^+.
\end{equation*}
Myonen haben eine Masse von \SI{105.7}{\MeV} und durchdringen Materie aufgrund ihrer vergleichsweise hohen Masse und der fehlenden starken Wechselwirkung sehr leicht.
Mit $\tau=\SI{2}{\micro\second}$ haben Myonen nach den Neutronen die längste Lebensdauer aller instabiler Elementarteilchen.
\newpage
\subsection{Tauonen}
Wir die Schwerpunktsenergie weiter erhöht, so können $\tau^-\tau^+$-Paare erzeugt werden.
Diese sind aber mit $\tau=\SI{3e-13}{\second}$ deutlich kurzlebiger als ihre Myon-Kollegen, weshalb sie in Detektoren nur über ihre Zerfallsprodukte nachgewiesen werden können:
\begin{figure}[h!]
	\centering
	\includegraphics[width=.8\textwidth]{./img/tauon-prod.pdf}
\end{figure}
Die dabei entstehenden Neutrinos sind in der Regel nicht nachweisbar.
Die Masse der Tauonen beträgt \SI{1777}{\GeV}.

\subsection{Bhabha-Streuung}
\begin{figure}
	\centering
	\includegraphics[width=.7\textwidth]{./img/bhabhaprocesses.pdf}
	\caption{Links: $\el\pos$-Streuung, Rechts: Paarvernichtung/-erzeugung}
	\label{fig:bhabha}
\end{figure}
\begin{figure}
	\centering
	\includegraphics[width=.7\textwidth]{./img/bhabhacross.pdf}
	\caption{Wirkungsquerschnitt der $\el\pos$-Reaktionen}
	\label{fig:crossbhabha}
\end{figure}
Die Erzeugung geladener Leptonenpaare kann näherungsweise als rein elektromagnetischer Prozess angesehen werden.
Die elastische Streuung $\el\pos\rightarrow\el\pos$ nennt sich auch \textbf{Bhabha-Streuung}.
Sowohl die $\el\pos$-Paarvernichtung mit anschließender Erzeugung eines $\el\pos$-Paares aus einem virtuellen Photon, als auch die bloße Streuung von Elektron und Positron aneinander führen dabei zum gleichen Endzustand, wie in \autoref{fig:bhabha} zu sehen ist.
Daher müssen bei der Berechnung des Wirkungsquerschnittes beide Feynman-Amplituden addiert werden.
Die Myon-Antimyon-Erzeugung ist einfacher zu berechnen und der differenzielle Wirkungsquerschnitt lautet
\begin{equation*}
	\diff[\sigma]{\Omega} = \frac{(\alpha\hbar c)^2}{4s}\cdot(1+\cos^2\theta).
\end{equation*}
Die übrigen $\el\pos$-Reaktionen werden auf diesen (totalen) Wirkungsquerschnitt normiert.

In \autoref{fig:crossbhabha} sind die Wirkungsquerschnitte beider besprochenen Reaktionen abgebildet.
Ist die Schwerpunktsenergie ausreichend hoch, sodass die Ruhemassen der beteiligten Teilchen vernachlässigt werden können, so ergibt sich ein nahezu identischer Wirkungsquerschnitt für Tauonen und Myonen.
Dieses Phänomen wird als \textbf{Leptonenuniversalität} bezeichnet und meint, dass sich Elektronen, Myonen und Tauonen abgesehen von ihrer Masse gleich verhalten.

Da die obige Formel für den Wirkungsquerschnitt die Messdaten gut reproduziert, ist der Formfaktor dieser Reaktion Eins.
Dies bedeutet, dass diese Leptonen in der Tat punktförmige Teilchen sind.

\subsection{Resonanzen}
Betrachtet man den Wirkungsquerschnitt für die Erzeugung von Myonenpaaren und Hadronen in $\el\pos$-Streuung, so findet man jeweils die $\nicefrac{1}{s}$.
Dieser Funktion sind im hadronischen Ausgangskanal ausgeprägte Maxima überlagert, die man \textbf{Resonanzen} nennt.
Man kann diesen Resonanzen wie bei ordinären Teilchen eine Masse und Quantenzahlen zuweisen.
\autoref{fig:resonances} zeigt verschiedene Resonanz-Peaks.

\subsection{Nichtresonante Erzeugung von Hadronen}
Selbstverständlich können auch zwischen den Peaks Quark-Antiquark-Paare erzeugt werden.
An die Primärquark-Antiquarks lagern sich dabei weitere Quark-Antiquark-Paare an und es entstehen Hadronen.
Diesen Vorgang nennt man \textbf{Hadronisierung}.
Natürlich können dabei nur Teilchen hadronisiert werden, deren Masse kleiner als die halbe zur Verfügung stehende Schwerpunktsenergie ist.
Im Gegensatz zu Leptonen tragen die Quarks nicht eine volle Elementarladung, sondern nur drittelzahlige Ladungen $z_\text{q}\cdot\text{e}$, die vom jeweiligen Quark-Flavor abhängig sind.
Da es ebenfalls 3 verschiedene Farbzustände gibt, geht in den Wirkungsquerschnitt noch der Faktor 3 mit ein.
Normiert man nun den Wirkungsquerschnitt des hadronischen Kanals mit dem Myonpaar-Wirkungsquerschnitt, so ergibt sich das Verhältnis
\begin{equation*}
	R=\frac{\sum_q\sigma(\el\pos\rightarrowq\bar{q})}{\sigma(\el\pos\rightarrow\mu^-\mu^+)} = 3\cdot\sum_qz_\text{q}^2,
\end{equation*}
welcher experimentell sehr leicht durch $\el\pos$-Kollisionen zugänglich ist.
Summiert werden bei obiger Formel nur die Ladungen der Quarks, für deren Erzeugung die Schwerpunktsenergie ausreicht.
Diese Messung ist eine eindrucksvolle Bestätigung dafür, dass es genau 3 Farben gibt, denn bei obiger Herleitung wurde vorausgesetzt, dass es 3 Farben gibt, was die Messdaten hervorragend reproduziert.
