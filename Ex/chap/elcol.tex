\chapter{Elektron-Positron-Kollisionen}
\section{Allgemeines}
In der Elektron-Positron-Annihilation können alle schwach und elektromagnetisch Wechselwirkenden Teilchen erzeugt werden.
Austauschbosonen sind in diesem Fall entweder das Photon $\gamma$ oder das $Z^0$-Boson.
Da Neutrinos elektrisch ungeladen sind, können Neutrino-Antineutrino-Paare nur durch den Austausch eines $Z^0$ erzeugt werden.

In einem Speicherring mit kollidierenden Teilchen gleicher Energie beträgt die maximale Schwerpunktsenergie $\sqrt{s}=2E$, womit alle Teilchen-Antiteilchen-Paare der Massen $2m = \nicefrac{\sqrt{s}}{c^2}$ erzeugt werden können.
Anders als bei Fixed-Target-Experimenten steht in Collider-Experimenten die gesamte Schwerpunktsenergie zur Verfügung, während im anderen Fall nur $s\approx 2MEc^2$ genutzt werden können, die Schwerpunktsenergie wächst also nur mit der Wurzel der Strahlenergie.

\section{Erzeugung von Leptonenpaaren}
\subsection{Myonen}
Sie sind die leichtesten Teilchen, die in $\el\pos$-Reaktionen erzeugt werden können:
\begin{equation*}
	\el + \pos \rightarrow \mu^- + \mu^+.
\end{equation*}
Myonen haben eine Masse von \SI{105.7}{\MeV} und durchdringen Materie aufgrund ihrer vergleichsweise hohen Masse und der fehlenden starken Wechselwirkung sehr leicht.
Mit $\tau=\SI{2}{\micro\second}$ haben Myonen nach den Neutronen die längste Lebensdauer aller instabiler Elementarteilchen.

\subsection{Tauonen}
Wir die Schwerpunktsenergie weiter erhöht, so können $\tau^-\tau^+$-Paare erzeugt werden.
Diese sind aber mit $\tau=\SI{3e-13}{\second}$ deutlich kurzlebiger als ihre Myon-Kollegen, weshalb sie in Detektoren nur über ihre Zerfallsprodukte nachgewiesen werden können:
\begin{figure}[h!]
	\centering
	\includegraphics[width=.8\textwidth]{./img/tauon-prod.pdf}
\end{figure}
Die dabei entstehenden Neutrinos sind in der Regel nicht nachweisbar.
